\chapter*{Abstract}{
\addcontentsline{toc}{chapter}{Abstract}
\pagestyle{plain} % this line is needed to remove CONTENTS from page headers

\begin{center}
%TODO change the title - Should be in Title Case (This is Title Case)
\large{\bfseries{Explainable AI Architectures: Methods, Applications, Examples, and Results}}

by

%%TODO Change the Author
\bfseries\MakeUppercase{Paul C. Whitten}
\end{center}

Artificial Intelligence (AI) has achieved results surpassing human capabilities
in recognition tasks. Systems have widely integrated Machine Learning (ML), a
field of AI, achieving outstanding results. Despite their success, ML systems
are opaque boxes and remain untrusted by humans. To establish better trust,
decisions in automated systems must be explainable.

This research approaches the problem of improving ML's explainability by
introducing two explainable methods and architectures. The first is a
property-based explainable method that provides the rationale for decisions
through reasoning about explainable properties. The second is a case-based
explainable method in which related training cases justify decisions. Following
the methods results in producing the components to construct the explainable
architectures. Herein, the explainable architectures are applied to the common
research problem of recognizing handwritten characters and a more focused
application of detecting hardware trojans in gate-level netlists.

As the methods are applied, their strengths and tradeoffs emerge. The
property-based explainable method benefits representational learning
applications with data that has many dimensions, while the case-based
explainable method excels in applications with few dimensions, requires low
effort to implement, and higher relative recognition accuracy than the
property-based method.

Applying the property-based explainable architecture to Modified National
Institute of Standards and Technology handwriting database achieved a
recognition accuracy of 97.3\% with an average explainability of 76.8\%. The
case-based architecture had an accuracy of 98.4\% with a correspondence factor
of 97.9\% on MNIST. The case-based architecture produced a 97.4\% correspondence
factor for training cases on Trust-Hub hardware trojans.

In addition to the methods, this work presents several contributions in the form
of concepts, techniques, and metrics. Contributions include explainable property
transforms, probabilistic voting, an explainable knowledgebase, an explainable
training index, techniques for unbalanced training data, handling ambiguous
training data, indicating cases where the system may fail, and an unexplainable
component to mitigate performance loss. Metrics presented include ($E_{PARS}$)
the effectiveness of decision systems to recognize a class, an explainability
metric, and the correspondence of training samples to the prediction. Leveraging
the explainable methods and contributions will improve human trust in and
adoption of decision-making systems.

\thispagestyle{plain} % this line is needed to remove the CONTENTS from the final abstract page header
}
