\chapter{Methods} \label{ch:methods}
% remove below this line and add your background

This chapter presents two explainable methods and architectures. Each method
provides a systematic means of assembling the explainable architectures from
components to compose an explainable system. The methods uniquely address
data dimensionality, imbalanced training data, and tradeoffs between prediction
performance and explainability. The strengths and weaknesses of the two methods
are outlined before detailing the methods.

The first property-based method works well for high-dimensional data such as
images in representational learning. When applied to a lower dimensional space,
such as a minimal number of features, the property-based method suffers from low
explainability. Low explainability in the property-based method will be
illustrated later in Chapter \ref{ch:trojan_results} when using only five
features in hardware trojan detection. The second case-based method works well
in data with lower dimensions.

The prediction performance between the two methods can differ. With the
property-based method, there is a loss associated with the property
transformations. Some property transforms have poor performance for particular
classes. The poor performance will detract from overall prediction performance.
The property-based system can mitigate some of the performance loss by
introducing an unexplainable component. While gaining prediction performance,
however, the unexplainable component detracts from explainability. 

The case-based system does not suffer from the same loss of prediction
performance. The ML model used in the case-based method is relatively
independent and can use the best-performing models directly on the application
data without the loss associated with property transformations. As a result, the
prediction performance of the case-based system is competitive with the latest
ML methods.

Figure~\ref{fig:method_concepts} represents this work's explainable artificial
intelligence (XAI) methodologies, highlighting related concepts, techniques,
metrics, similarities, and differences. The property-based and case-based
methods are represented in the distinct branches, extending from the root XAI
node. Techniques used in the property-based method include transformations,
voting, knowledgebase, and unexplainable components. Techniques used in the
case-based method include KNN and a training index. Metrics introduced in the
property-based method include effectiveness, confidence, and explainability. The
case-based method presents correspondence and balance factor metrics. The
methods' differences in dataset strengths are shown in the two
\lq{dimensionality}\rq nodes. The property-based method performs better with
high-dimensional datasets such as images, while the case-based method favors
datasets with lower dimensions. Figure \ref{fig:method_concepts} also highlights
similarities between the methods, such as techniques for handling imbalanced
training data, bother leveraging storage, and having an explainable processing
stage.

\begin{figure}[H]
    \centerline{\includegraphics[width=13.5cm, alt={A concept tree of the explainable methods}]{./images/method-intro-dot.png}}
    \caption{A tree representing the explainable method concepts (such as explainable processing, unexplainable component, explainable property, and property transforms), techniques (such as KNN and voting), and metrics (like correspondence, balance factor, effectiveness, confidence, and explainability). Storage and dimensionality of the datasets are also reflected.}
    \label{fig:method_concepts}
\end{figure}

% In the property-based method, explainable property transforms change the
% input data and have been observed to typically diminish recognition accuracy. In
% the property-based method, high performing unexplainable components are
% introduced to improve prediction results beyond.  This comes at a cost to
% explainability and reduces prediction accuracy from the highest performing
% unexplainable models.


% This chapter outlines the contributions of this work by presenting the
% methodologies, architectures, as well as the experimental and computational
% methods used in developing decision-making architectures that have the ability
% to explain results. Initially the explainable property-based methodology is
% presented.  The explainable property-based methodology is then detailed as it is
% applied to explainable detecting handwritten characters.  A method and
% architecture for case-based explanations follows. Both methodologies are then
% applied to explainable detection of hardware trojans.

%TODO: do we want to list the four IEEE papers and conference presentations
%or is that better for the presentation?


\section{Definitions and Terms}

The following notions are defined here as relevant to the methodologies. A
\textit{recognition system} is a type of AI used to identify and classify
inputs. \textit{Application input} to the recognition system represents an
element or sample, which may be of various classes. A \textit{class} is a type,
category, or group of elements that a recognition system is designed to
identify. In supervised learning, \textit{classes} are typically distinguished
as unique labels. For example, a handwriting recognition system input may be a
28x28 grayscale image representing a character. A class would be one of the
characters the recognizer is designed to identify, such as a handwritten zero.

A \textit{model} is a program implementing an AI algorithm that has been taught
to recognize patterns or make decisions without outside intervention. In the
case of supervised learning with an FFNN, the \textit{model} consists of the
parameters, hyperparameters, and any code that may facilitate data processing of
input samples.

% The term problem domain is used to refer to the context in which the problem is
% approached. It assumes some basic understanding of the subject matter concerning
% the problem.

%TODO lsit metrics used
%\section{Metrics}

% This section reviews metrics presented in this method.
% 

\section{Explainable Property-Based Methodology}

The explainable property-based methodology is a seven-step process conceived by
looking at how humans may explain why they recognize a sample as a particular
handwritten character class. A notion of explainable properties of the samples
emerged. The intent is to reason about the system's decisions regarding the
explainable properties.

\begin{figure}[H]
    \centerline{\includegraphics[width=17cm, alt={A flowchart of the property-based explainable method}]{./images/pbe-flow.png}}
    \caption{Flow of the property-based explainable methodology.}
    \label{fig:prop_xai_flow}
\end{figure}

An enumeration of the seven steps of the explainable property-based methodology
follows. The flow diagram in Figure~\ref{fig:prop_xai_flow} depicts the seven
steps of the method. The methodology steps are in blue in
Figure~\ref{fig:prop_xai_flow}. The steps begin at the start and proceed down
the bold, directed edges. At each stage that produces an artifact, that artifact
is depicted by a dotted directed edge. The methodology is intended to be
iterative with feedback. At any time, past steps may be revisited to refine the
result. The test phase in the flow is also used to assess overall performance
and, if needed, revisit previous steps to make improvements. As the methodology
is explained, more detail will be provided for each of the steps.

\begin{enumerate}
    \item Data processing
    \item Discover explainable properties
    \item Define and implement property transformations
    \item Transform training data
    \item Train ML models for each transform
    \item Build a knowledgebase
    \item Devise and implement a voting scheme
    \item Build an explanation routine
\end{enumerate}


This methodology is meant to be used with supervised ML techniques. The ML
algorithms employed may vary based on preferences, desired performance, or
available compute. When examining the methodology regarding the AI taxonomies
posed in Section \ref{sec:ai}, the explainable property-based methodology
produces a weak (narrow) and limited memory AI. The resulting system is narrow
as it addresses a particular problem. It is categorized as limited memory
because it uses a knowledgebase to store historical information from long-term
training.

\subsection{Step One - Data Processing} % step 1

Data processing involves curating data used to train and test the system.
Effective data curation can improve model performance. Some common data
processing tasks involve the initial acquisition or collection of the data. The
data may need to be cleaned where inconsistencies or ambiguities are removed.
Transformation of the data into new formats may be required for learning
techniques. Data processing may involve feature engineering, where initial
patterns may be used to improve model performance.

\subsection{Step Two - Method Discover Explainable Properties} % step 2

An explainable property is an attribute of an input sample that may
differentiate between classes and provide a rationale for a classification
decision to a user. An explainable property need not be present in all classes,
and it need not distinguish all classes. Some samples of explainable properties
for handwritten characters are common geometric features of characters such as
lines, curves, enclosed regions, inflection points (corners), and crossings and
endpoints.

The second step of the explainable property-based methodology is to examine the
problem and discover explainable properties. Discovering explainable properties
is done by looking for sample attributes that help distinguish between classes.
It could be accomplished by manually browsing through samples in training data
or by some automated means, such as clustering among classes and looking for
similarities of elements between clusters. Principal Component Analysis could be
another means of discovering properties, providing the Principal Components can
be interpreted and explained. After completing this step, there may be $n$
properties to consider, as shown in Figure~\ref{fig:prop_xai_flow} with the
Explainable Properties artifacts.

%TODO: do we need to put PCA in the background?

\subsection{Step Three - Property Transformations} % step 3


\begin{figure}[h]
    \centering
    \includegraphics[width=9.0cm, alt={A diagram of a property transform}]{./images/property_transform.png}
    \caption{A property transformation modifies the input to produce output that highlights an explainable property.}
    \label{fig:property_transform_ex}
\end{figure}

The third step of the methodology is to define and implement property
transformations from the $n$ explainable properties identified in the first
step. As shown in Figure~\ref{fig:property_transform_ex}, a property
transformation is a function $T_p$ to modify an input sample, $i$, to bring out
an explainable property, $p$, in the resulting output. A property transformation
aims to highlight or exemplify explainable properties in the input. Transforms
may be known feature detection and extraction algorithms related to explainable
properties.

Algorithms should be identified or implemented to transform input samples. In
the case of existing algorithms serving as transforms, experimentation should be
performed to find suitable parameters for the algorithm for the sample input.
The association between explainable properties and property transforms will be
maintained in the knowledgebase, which is discussed later, to aid in
explainability. 

Some explainable properties may not easily be paired with transforms, and others
may have multiple means of transforming the input to exemplify the properties in
the output. At the end of this step, there may be $m$ Transformation Routines as
depicted in Figure~\ref{fig:prop_xai_flow}.

\subsection{Step Four - Transform Training Data} % step 4

After implementing property transformations, the fourth step in the methodology
is to transform the training data, storing the transformed data for later
training. Transforming training data involves feeding each training sample
through the property transforms and storing them as depicted in
Figure~\ref{fig:property_trans_training}. A training element, in this case, an
image, $i$, is distributed to each of the $m$ property transforms. The resulting
output, Transformed Training Data in Figure~\ref{fig:prop_xai_flow}, is stored
in a training store for that property transformation. Naturally, it will be
important to preserve the label of training samples. In the methods here, the
order of training samples was preserved, so labels remained unchanged.

\begin{figure}[h]
    \centering
    \includegraphics[width=12.0cm, alt={A diagram illustrating the transformation of training data}]{./images/property-trans-training-db-dot.png}
    \caption{Transforming and storing training data.}
    \label{fig:property_trans_training}
\end{figure}

While transforming a single element by the $m$ transforms is not costly,
transforming many training samples by $m$ transforms can take significant time.
Exploiting parallelism in transforming the training data may result in a
significant speedup.

\subsection{Step Five - Train ML Models} % step 5

The fifth step in the methodology involves producing an ML model for each
property transformation. Producing an ML model is accomplished by selecting an
appropriate ML algorithm and architecture and then training using the particular
transformed training data for that property transform to produce the model. This
step can be an iterative process involving training, testing, and adjusting the
model hyperparameters. It may be necessary to revisit and reconsider property
transform settings.

\begin{figure}[h]
    \centering
    \includegraphics[width=13.0cm, alt={A diagram illustrating transforms connected to inferece engines in the property-based architecture}]{./images/property-ml-dot.png}
    \caption{Property transform ML models connected to property transforms begin to form the initial phases of the explainable property-based architecture.}
    \label{fig:property_ml}
\end{figure}

Other areas of this work refer to the individual ML models associated with a
property transform as an inference engine (IE). Figure~\ref{fig:prop_xai_flow}
depicts the results of training models as a set of Inference Engines. At this
point in the explainable methodology, a set of components can be used to begin
constructing the explainable architecture, as shown in
Figure~\ref{fig:property_ml}. The property transforms can be connected to a
sample input function, and the output of the property transformations can be
connected to each property transform's ML model to perform $m$ inference on an
input sample. The inferences performed by each of the inference engines can be
thought of as a local opinion based only upon that property transformation.

\subsection{Step Six - Build a Knowledgebase} % step 6

The previous steps of the methodology provided the components to construct an
architecture that can perform $m$ inferences on input, as observed in
Figure~\ref{fig:property_ml}. The following steps will allow those local
inferences to be used to make a global decision. The knowledgebase is key to
making the appropriate global decision, reasoning intelligently about the
decision, and forming the explanation.

This phase of the explainable property-based methodology involves constructing
the knowledgebase that will be used as the explainable system performs
inferences and explanations. The knowledgebase contains data necessary for
making important decisions on the potentially conflicting local property
transform inferences and details about properties and transforms that will help
explain decisions.

A key component stored in the knowledgebase is the effectiveness metrics, which
are covered in Section \ref{voting_schme}. The effectiveness of each transform
in predicting each class is calculated and stored in the knowledgebase. The
voting scheme relies on effectiveness to weight, potentially conflicting
opinions.

Test data is fed back through the explainable architecture to produce summary
results of inferences on the training data. Ideal summary data to store in the
knowledgebase for the inferences are the confusion matrices for each ML model.
This step may be best combined with the training step to take out-of-sample
summary results of training. Experiments used raw training data, test data, and
k-fold cross-validation without observing differences between the methods.
Little difference between the methods is likely because the summary data is used
to compare ML models, and any optimism resulting from in-sample data had little
effect. 

The knowledgebase will require information mapping each property transformation
to its related property. Relating properties to transforms is necessary in
formulating explanations. The relation between transforms and properties may be
accomplished by using a descriptive string linking each transformation to its
property.

For the explainable property-based architecture to communicate when there is
potential for error, it is also worthwhile to store test results in the form of
a confusion matrix in the knowledgebase. This step can be completed after the
voting phase.

\subsection{Step Seven - Voting Scheme} % step 7
\label{voting_schme}

Voting involves the selection of a global decision when faced with conflicting
opinions from property inferences. In many cases, the inference engines' local
opinions will not agree. The seventh step in the methodology is to implement a
voting scheme to select among the inference engine opinions. To improve
explainability and the system's performance, careful selection of a voting
algorithm is required.

A key concept to making a good voting decision is the effectiveness of an
inference engine in making a decision. Effectiveness is obtained from summary
information in the knowledgebase from processing test data. The effectiveness of
an inference engine, $j$, to correctly recognize an item of class $c$ is
expressed as $E(j,c)$. A good choice of effectiveness metric depends on how
balanced the data is. Good choices of effectiveness metrics are those resilient
to data imbalance from Section \ref{sec:perf_metrics}. Once an appropriate
effectiveness metric is selected, the effectiveness of each inference engine for
each class is calculated from the summary metrics, and effectiveness is stored
in the knowledgebase.

The voting scheme posed involves considering the opinions (votes) of the
inference engines in calculating the confidence $Conf(c)$ of a class, $c$ that
was voted for by an inference engine, as the ratio of the weighted effectiveness
of that class, $WE(c)$, over the weighed effectiveness of all classes with
votes.

The weighed effectiveness of a class, $c$, is given by $WE(c)$ in
\eqref{eq:weighted_effectiveness} the sum of the effectiveness for the $j$
inference engines that voted for $c$. The selection of the effectiveness metric
$E$ is addressed In Section \ref{effectiveness_metrics}. 

\begin{equation}\label{eq:weighted_effectiveness}
    WE(c)=\sum_j E(j, c)
\end{equation}

The confidence, $Conf(c)$, for a class, $c$, is then calculated in
\eqref{eq:conf} as the weighted effectiveness of the class $c$ over the sum of
weighted effectiveness for the classes $k$, that each inference engine voted.
Confidence of a class, $c$, will be $0 \leq Conf(c) \leq 1.0$. The sum of the
confidence of all classes suggested will equal $1.0$.

\begin{equation}\label{eq:conf}
    Conf(c)=\frac{WE(c)}{\sum\limits_kWE(k)}
\end{equation}

The output of this phase is a Voter, as shown in Figure~\ref{fig:prop_xai_flow}.
The Voter selects the class $c$ with the highest confidence as the winner. The
Voter may also indicate the alternatives considered with their confidence
metrics.

\subsubsection{Effectiveness Metrics}
\label{effectiveness_metrics}

It is not enough to know that an inference engine is effective overall.
Effectiveness must be examined at the class level. The effectiveness of a
particular inference engine, $j$, for a class, $c$, is given as $E(j,c)$. The
particular effectiveness metric is selected based on the optimal metric based on
the data imbalance.

In a dataset with multiple classes, the metrics are calculated as one versus all
other classes. The one versus others strategy imposes a significant imbalance
between the single class and others with many datasets, including those used in
this work. Some metrics are sensitive to imbalanced data and produce misleading
results. Imbalanced data is often the case in recognition problems with multiple
classes. It is important to have metrics resilient to such imbalances.

Particular metrics from the literature that were examined and evaluated as
effectiveness in the explainable property-based methodology include Accuracy,
Recall, Specificity, Precision, F1-Score, Cohen's Kappa, Matthews Correlation
Coefficient, Balanced Accuracy, and the Area Under the Curve of the Receiver
Operator Characteristic (AUC).

In addition to evaluating metrics from the literature, some metrics were
combined and compared experimentally. The combination of metrics found to have
the best performance as per-class effectiveness was the product of Precision
($P$), Accuracy ($ACC$), Recall ($R$), and Specificity ($S$) as shown in
\eqref{eq:epars} as the new effectiveness metric, $E_{PARS}$.
Equation~\eqref{eq:epars_expansion} shows the expansion of $E_{PARS}$ in terms
of  true positives $(TP)$, true negatives $(TN)$, false positives $(FP)$, and
false negatives $(FN)$\cite{whitten24icmi}.

\begin{equation}\label{eq:epars}
    E_{PARS} = P \cdot ACC \cdot R \cdot S
\end{equation}

\begin{equation}\label{eq:epars_expansion}
    %E_{PARS} = P \cdot ACC \cdot R \cdot S  \\
    %&= {\scriptstyle \frac{TN \cdot TP^2(TP+TN)}{(TN+FP)(TP+FP)(TP+FN)(TP+TN+FP+FN)} }
\frac{TN {\cdot} TP^3+TN^2 {\cdot} TP^2}{(TN{+}FP)(TP{+}FP)(TP{+}FN)(TP{+}TN{+}FP{+}FN)}
\end{equation}

\subsection{Step Eight - Property-Based Explanation Routine} % step 8

The previous steps in the explainable property-based methodology allowed a
system to be constructed that can make decisions. The eighth and final step in
the methodology assembles an explanation from the data in previous steps based
on the explainable properties.

Particular opinions from the inference engines are important in composing the
explanation. The classes and their confidence may be explained by attributing
the decision based on the inference engines and associated properties that voted
for a class. The descriptive strings stored in the knowledgebase may be used to
formulate a textual explanation. The effectiveness of each inference engine can
be leveraged to rank or quantify the contribution from the properties in
providing an explanation.

The algorithm for composing the explanation in the explainable property-based
architecture is referred to as XAI and the output of this step is shown in
Figure~\ref{fig:prop_xai_flow} as the XAI block. The XAI block may be
implemented as a program, service, or subroutine requiring inference engine
opinions, confidence from the voting scheme, and access to the knowledgebase for
effectiveness and descriptive strings for property transforms.

\subsection{Improving Results by Adding Unexplainability}

In the property-based system, explainability is provided by explainable
properties, used to justify a local decision in plain terms to a user. The
performance of some inference engines is acceptable, while others are lacking.
Poor results in some inference engines are because the transforms omit
information in the original input. The recognition performance of the system may
be improved by adding traditional ML models as new inference engines trained on
the unaltered input. These new inference engines would have outstanding
recognition performance but no explainability, like traditional opaque,
unexplainable ML models. Because the new inference engines are not associated
with an explainable property and, therefore, are considered unexplainable, a
means of quantifying explainability is desired.

With the introduction of unexplainable inference engines, each inference engine,
$j$, is assigned an explainability metric $X_j$, where $ 0 \leq X_j \leq 1$. An
$X_j$ close to zero would signify an unexplainable inference engine, while an
$X_j$ near one would indicate an explainable inference engine. Examples in this
work use $X_j=1$ for explainable and $X_j=0$ for the unexplainable inference
engines.

\begin{equation}\label{eq:explainability}
Ex(c)=\frac{\sum{E(j,c)X_j}}{\sum{E(j,c)}}
\end{equation}

The explainability of a decision for class $c$ is given by $Ex(c)$ in
\eqref{eq:explainability} where the numerator is the sum of the product of the
explainability metric and effectiveness for the inference engines that voted for
class $c$. The denominator is the sum of effectiveness for inference engines
that voted for class $c$. Using the effectiveness in this way weights the
explainability based on how much it contributed to the decision.

\subsection{Explainable Property-Based Architecture}

This next section poses an explainable property-based architecture based on the
property-based explainable methodology. Figure~\ref{fig:xai_arch} depicts the
explainable architecture, composed of produced artifacts from the steps of the
property-based methodology in Figure~\ref{fig:prop_xai_flow}. The architecture
in Figure~\ref{fig:xai_arch} has explainable and unexplainable contributions.
Consider each horizontal element in the architecture incident on the
Decision-Making Process as a Pre-Decision Flow (PDF) of the explainable system.
An explainable PDF would contain one Property Transform and one Inference Engine
(IE) related to an explainable property. An unexplainable PDF would consist of
only one IE.  An explainable property and transform are lacking for an
unexplainable PDF. The $1$ through $n$ explainable PDFs in the architecture are
depicted, while the single unexplainable PDF is represented at the bottom as
$IE_u$. Again, explainable PDFs are related to properties that contribute to the
explainability of the system. Often, explainable PDFs provide mediocre
recognition results compared to an IE trained against untransformed data. The
mediocre performance is often because a property may not pertain to each class.
One such example is the ellipse property does not apply to a well-formed digit
one.

\begin{figure}[H]
    \centering
    \includegraphics[width=15.0cm, alt={The property-based explainable architecture consisting of transforms, inference engines, a voter or decision making process, and an XAI block}]{images/pbe_arch_general.png}
    \caption{Property-Based Explainable AI Architecture}
    \label{fig:xai_arch}
\end{figure}

In this architecture, the unexplainable PDF is the untransformed input image.
The unexplainable flow contributes excellent recognition performance but deters
from the explainability of the system since the IE is considered an opaque box
without an explainable property for the justification of a decision. An
explainability metric is introduced to quantify the impact of unexplainable PDFs
on decisions in \eqref{eq:explainability}.

In Figure~\ref{fig:xai_arch}, input flows into the system at the left. The
initial stage of the architecture for explainable PDFs is the Transform phase.
The input image is transformed according to the explainable property in the
Transform phase. Transforms emphasize particular explainable properties in the
input and are essential to explainability.

The second stage of the architecture is the Property Inferencing stage. In
Property Inferencing, IEs are used to make decisions based on local data. An IE
may use an appropriate ML algorithm. IEs in this work are MLP, SVM, CNN, or
ResNet models. Explainable IEs are trained on transformed training data. The
unexplainable IE, $IE_u$, is trained on untransformed input. Metrics from
training are stored in a knowledgebase (KB) and used by Decision-Making and XAI
as depicted at the bottom of Figure~\ref{fig:xai_arch}.

Local decisions from Property Inferencing are passed to Decision-Making, where a
global decision is made and assigned a confidence. The Decision-Making process
may use a probabilistic voting scheme described in the methodology to make a
global decision.

The Explainability, XAI, phase in the explainable Architecture,
Figure~\ref{fig:xai_arch}, composes the textual rationale related to the
explainable properties contributing to decisions. This is done by constructing
phrases indicating the properties related to the opinions for each class that
received a vote. The explainability measure, $Ex(c)$ from
\eqref{eq:explainability}, is also determined in this phase of the system. The
measure indicates how explainable each decision for a class, $c$, is based on
the relation to an explainable property of the flows that suggested $c$. The
user is finally presented with the ranked recognition results, the confidence of
the decisions, the rationale for a decision, the level of explainability for
each decision, and how the architecture has historically failed for the current
predicß. Sample explainable results are presented in Section
\ref{ch:handwriting_results}.

\subsection{Communicating when the Explainable Property-Based Architecture Fails}

Referring back to Figure~\ref{fig:ml_now_xai_future}, one of the questions an
explainable AI should answer is, "When do you fail?" The knowledgebase in the
property-based explainable architecture can store the overall results from test
data. With the results in a confusion matrix, metrics such as the false
discovery rate (FDR) as given in \eqref{eq:fdr} indicate the rate of erroneous
predictions for a given class.

\begin{equation}\label{eq:fdr}
    FDR = \frac{FP}{TP + FP} 
\end{equation}

Table \ref{tab:emnist_digit_svm} shows a confusion matrix for an SVM-based
explainable system generated from running predictions on the EMNIST digit test
set. Suppose the confusion matrix was stored in the knowledgebase of an
explainable system. If that explainable system predicted an application input as
the digit nine, the confusion matrix may be used to indicate that about 4.2\% of
the time, the system is wrong when it predicts a nine. The value is obtained by
calculating the false discovery rate of the digit nine, $FDR_9$, which is:

\begin{equation}\label{eq:fdr_9}
    FDR_9 = \frac{FP}{TP + FP} = \frac{2+7+1+6+1}{385+2+7+1+6+1} =  \frac{17}{402} = 0.042
\end{equation}

\begin{table}[H]
    \centering
    \caption{SVM confusion matrix for EMNIST digits.}
    \label{tab:emnist_digit_svm}
    \renewcommand{\arraystretch}{1.3}
    \begin{tabular}{ll|c|c|c|c|c|c|c|c|c|c|}
        \multicolumn{2}{c}{}& \multicolumn{10}{c}{Predicted}\\
        & \multicolumn{1}{c}{} & \multicolumn{1}{c}{0} & \multicolumn{1}{c}{1} & \multicolumn{1}{c}{2}
        & \multicolumn{1}{c}{3} & \multicolumn{1}{c}{4} & \multicolumn{1}{c}{5} & \multicolumn{1}{c}{6}
        & \multicolumn{1}{c}{7} & \multicolumn{1}{c}{8} & \multicolumn{1}{c}{9} \\
        \cline{3-12}
        \multirow{10}{*}{{\rotatebox[origin=c]{90}{Actual}
        }} & 
        0 &     393 & 0 & 0 & 1 & 1 & 1 & 3 & 0 & 1 & 0 \\ \cline{3-12}
        &   1 & 0 & 395 & 3 & 0 & 2 & 0 & 0 & 0 & 0 & 0 \\ \cline{3-12}
        &   2 & 3 & 0 & 391 & 3 & 1 & 1 & 1 & 0 & 0 & 0 \\ \cline{3-12}
        &   3 & 2 & 0 & 1 & 390 & 1 & 1 & 0 & 1 & 2 & 2 \\ \cline{3-12}
        &   4 & 3 & 0 & 1 & 0 & 389 & 0 & 0 & 0 & 0 & 7  \\ \cline{3-12}
        &   5 & 0 & 0 & 2 & 3 & 1 & 389 & 1 & 1 & 2 & 1 \\ \cline{3-12}
        &   6 & 0 & 2 & 1 & 0 & 1 & 1 & 395 & 0 & 0 & 0 \\ \cline{3-12}
        &   7 & 0 & 0 & 1 & 1 & 1 & 0 & 0 & 390 & 1 & 6 \\ \cline{3-12}
        &   8 & 1 & 0 & 0 & 1 & 2 & 4 & 1 & 0 & 390 & 1 \\ \cline{3-12}
        &   9 & 3 & 0 & 0 & 0 & 6 & 0 & 0 & 4 & 2 & 385 \\ \cline{3-12}
    \end{tabular}
\end{table}

In addition to the FDR, the confusion matrix can be used to indicate which
classes are the most likely to be erroneously suggested for the current
prediction. Identifying such classes may be best approached using a threshold,
for example, $1\%$. When predicting a nine, the system may further suggest that
the digit four ($\frac{7}{402}=1.7\%$) and seven ($\frac{6}{402}=1.4\%$) are the
digits most often confused when predicting the digit nine.

\section{Case-Based Explainable Method}

The case-based explainable method is inspired by a desire to relate details of
what was learned from the training set to decisions made. In this method, cases
like an input encountered during training should support the system's decision.
When considering a new sample to a case-based explainable system, the system
produces like cases with an analysis indicating correspondence based on those
like cases.

\begin{figure}[H]
    \centering
    \includegraphics[width=18.0cm, alt={A flowchart of the case-based explainable method.}]{./images/case-based-xai-flow.png}
    \caption{Flow of the case-based explainable method.}
    \label{fig:case_xai_flow}
\end{figure}

Figure~\ref{fig:case_xai_flow} depicts the flow of the case-based explainable
method with steps in the flow as the green nodes connected by the heavy directed
edge. Where steps in the flow produce artifacts as output, those are depicted in
purple with dotted directed edges. An enumeration and then details of the steps
in the case-based explainable method follows:

\begin{enumerate}
    \item Data processing
    \item Train a machine learning model
    \item Construct a training index
    \item Devise a training index query scheme
    \item Implement a case-based explanation routine
\end{enumerate}

\subsection{Step One - Data Processing}

As with the property-based method and other general applications of machine
learning algorithms, the first phase involves preparing data for the learning
process. These methods will often be unique to the problem and are important,
since the methods applied in data processing on training data are also used with
a deployed AI system on the input.

\subsection{Step Two - Train ML Model}

A machine learning model is trained using data produced in step one, which is
intended explicitly for training the model. The model training will be done
traditionally (without transforms) and will, therefore, be unexplainable. The
output of this phase is a trained machine learning model capable of performing
the desired task, recognizing new input based upon its training.

\subsection{Step Three - Training Index}

The training data used to construct the ML model is processed to construct a
training index. The training index allows the training data to be rapidly
retrieved. Rapid retrieval may be accomplished by assigning identifiers (a
unique ID) to the distinct training samples and using a key-value storage
mechanism.

In addition to rapid lookup, the training index can be used to cache information
to reduce repeated calculations in the future. For example, distances can be
cached in the training index. Caching distances will significantly assist in
cases where the data dimensionality is low, and collisions are frequent. In the
case of high data dimensionality, the chance for collisions is low, and there is
less opportunity to use cached calculations.

\subsection{Step Four - Query Scheme} %TODO should this be called search?

The query scheme is crucial for this methodology as it allows similar cases from
the training set to be identified. The query scheme should allow a sample input
to be quickly matched to similar cases from training. One leading candidate for
a query scheme is clustering and nearest neighbor search. In the case of a
nearest neighbor search, the $k$ nearest neighbors in the training set are
retrieved by identifier.

\subsection{Step Five - Case-Based Explanation Routine}

A case-based explanation routine provides the textual justification linking an
input to similar cases from training. This routine takes the nearest neighbors
from the query and data from the training index to provide information on cases
similar to the input. Rather than merely providing neighboring samples, it is
crucial to provide metrics to quantify the samples in terms of the class
selected by the inference engine.

In addition to providing similar cases from training, the case-based explanation
routine provides metrics to support the data presented. In doing so, it is
important to account for the imbalance in the dataset in this method. Recall
from \eqref{eq:imbalance_ratio} the imbalance ratio is the number of majority
class instances over minority class instances. The desire is to consider
majority and minority instances relative to the imbalance. Overcoming imbalance
is accomplished by discounting the majority by the inverse of the imbalance
ratio. In \eqref{eq:balance_factor}, the balance factor, $bf$, of a class $c$ is
given as the inverse of the imbalance ratio if the class $c$ is the majority
class, otherwise $1.0$.

\begin{equation}\label{eq:balance_factor}
    bf(c) = 
    \begin{cases}
    \frac{1}{IR},& c = maj\\
    1.0,& c = min
    \end{cases}
\end{equation}

Nearest neighbor information presented to the user includes distance, $d$, from
the sample being considered, the classes of training samples, and references to
the samples, including potential origins from data processing. This originating
data for cases is important in providing transparency to the training data. In
low dimensionality datasets, neighbors tend to be grouped at coinciding
distances, $d$, from the sample where the number of distinct distances, $l$ of
the $k$ neighbors is commonly $l < k$.

As shown in \eqref{eq:weight_of_neighbors}, the neighbor data is used to
calculate the weight of neighbors, $WN$, of a class $c$ relative to the inverse
square of the distance, $d$, from the input sample considered. Each instance,
$i$, of a class $c$ in the $k$ neighbors is summed by taking the balance factor
from \eqref{eq:balance_factor} over the sum of the distance from that sample
$d_i$ and $1.0$. Because a training instance can be identical to a sample and
have a distance of zero, the $1.0$ is used in the denominator to prevent a
divide-by-zero situation. 

In \eqref{eq:correspondence}, the correspondence, $Corr$, of a class, $c$, is
given by the weight of that class over the weights of all classes in the $k$
neighbors. Correspondence of a class signifies how well the $k$ neighbor
instances agree with the class selection. Correspondence of a class $c$ will be
on the interval $0 \ge C(c) \le 1.0$. A high correspondence, near $1.0$, will
signify an agreement between the class selected and neighboring samples. In
contrast, a low correspondence, near 0, will signify little agreement between
the class selected and neighboring samples. The class with the high
correspondence is considered as the suggested class by neighbors. Alternatives,
the classes in the K neighbors with lower correspondence, are also presented.

\begin{equation}\label{eq:weight_of_neighbors}
    WN(c) = \sum_{i=1}^{c_i \in k} \frac{bf(c)}{(d_i+1.0)^2} 
\end{equation}

\begin{equation}\label{eq:correspondence}
    Corr(c) = \frac{WN(c)}{\sum_{j=1}^{c_j \in k}{WN(c_j)}} 
\end{equation}

The explanation routine presents the prediction of the trained ML model as a
class, the correspondence of that class based on the $k$ neighbors, and neighbor
data as the case-based justification for accepting or seeking an alternative. In
the rare event that the neighbors select a different class from the ML model,
the explanation routine will indicate the conflict.

\subsection{Case Based Explainable Architecture}

\begin{figure}[H]
    \centering
    \includegraphics[width=14.0cm, alt={The case-based explainable architecture consisting of a trained ML model, a training index, a query routine, and an explanation routine}]{./images/case-based-xai-arch.png}
    \caption{A case-based explainable architecture.}
    \label{fig:case_xai_arch}
\end{figure}

A case-based explainable architecture is presented in
Figure~\ref{fig:case_xai_arch} using the trained machine learning model, a
training index, and an explanation routine as artifacts produced from the flow
in Figure~\ref{fig:case_xai_flow}. Input samples are fed in at the left. The
trained machine learning model contributes a prediction to the system. The
training index efficiently retrieves cases and supporting data used to justify
the decision. The explanation component organizes the decision, correspondence
metric, and $k$ neighbors as supporting data to justify the decision.
