\chapter{Handwriting Results} \label{ch:handwriting_results}
% remove below this line and add your concluding remarks

This chapter presents the results of applying the property-based methodology to
images of handwritten characters.  Overall MNIST results are first presented
followed by detailed explainable examples.  A summary of EMNIST results will be
followed by detailed explainable results of handwritten characters. 

\section{Property-Based Explainable Results on Handwritten Characters}

\subsection{MNIST Results}
\label{sec:mnist_results}

Four types of inference engines were used in the property-based methodology.
Inference engine types included:

\begin{itemize}
    \item Multilayer perceptrons (MLP)
    \item Support vector machines (SVM)
    \item Convolutional neural networks (CNN)
    \item Deep residual learning (Resnet50)
\end{itemize}

Table \ref{tab:mnist_accuracy_results} outlines the overall recognition accuracy
observed with arrangements of the various inference engine types used.  Each row
in the table represents inference engine types as indicated in the first column.
The second column, labeled 1 Unexplainable, provides the baseline performance of
a single ML model in a representational learning configuration acting on
untransformed application input.  The SVM is the poorest performing at $97.9\%$
accuracy. The MLP is next with with $98.3\%$ accuracy.  The CNN baseline performs the
best overall with a $99.4\%$ accuracy followed by Resnet50 with $98.9\%$ accuracy.

The third column of Table \ref{tab:mnist_accuracy_results} presents the accuracy
results of the property based explainable architecture with ten explainable
property transformations. Again, SVM has the lowest accuracy, $95.4\%$, with the
MLP performing slightly better at $96.2\%$ recognition accuracy.  The
explainable SVM system is only $2.5\%$ behind the unexplainable baseline.  The
Resnet50 system performs best in this configuration with only a $1.3\%$ drop
from the unexplainable Resent50 benchmark.  The explainable CNN sees a $2.1\%$
decline from the unexplainable benchmark.

The final column in Table \ref{tab:mnist_accuracy_results} illustrates the
improvement on accuracy of adding one unexplainable flow to the system.  Observe
that many of the systems approach the recognition accuracy of the benchmark,
especially Resnet50.

\begin{table}
    \renewcommand{\arraystretch}{1.3}
    %\centering
    \caption{Accuracy of models and architectures on MNIST}
    \begin{center}
    %\resizebox{\columnwidth}{!}{%
    \begin{tabular}{|l|c|c|c|}
        \cline{2-4}
        \multicolumn{1}{c|}{} & \multicolumn{3}{c|}{Architecture} \\
        \hline
        Model Type & 1 Unexplainable & 10 Explainable & \parbox[t]{3.3cm}{10 Explainable\\1 Unexplainable}\\
        \hline
        \hline
        MLP & $98.3$ & $96.2$ & $97.9$ \\
        \hline
        SVM & $97.9$ & $95.4$ & $97.3$ \\
        \hline
        CNN & $99.4$ & $97.3$ & $98.7$ \\
        \hline
        Resnet50 & $98.9$ & $97.6$ & $98.8$ \\
        \hline
    \end{tabular}
    %}
    \end{center}
    \label{tab:mnist_accuracy_results}
\end{table}

\begin{table}
    \renewcommand{\arraystretch}{1.3}
    %\centering
    \caption{Average explainability of the winning choice for models and architectures on MNIST}
    \begin{center}
    %\resizebox{\columnwidth}{!}{%
    \begin{tabular}{|l|c|c|c|}
        \cline{2-4}
        \multicolumn{1}{c|}{} & \multicolumn{3}{c|}{Architecture} \\
        \hline
        Model Type & 1 Unexplainable & 10 Explainable & \parbox[t]{3.3cm}{10 Explainable\\1 Unexplainable}\\
        \hline
        \hline
        MLP & $0.0$ & $100.0$ & $67.2$ \\
        \hline
        SVM & $0.0$ & $100.0$ & $76.8$ \\
        \hline
        CNN & $0.0$ & $100.0$ & $75.5$ \\
        \hline
        Resnet50 & $0.0$ & $100.0$ & $69.9$ \\
        \hline
    \end{tabular}
    %}
    \end{center}
    \label{tab:mnist_explainability_results}
\end{table}

Table \ref{tab:mnist_explainability_results} presents the configurations from
Table \ref{tab:mnist_accuracy_results} but instead of recognition accuracy, the
average explainability value of the winning selection is given as a percentage.
The benchmark values are completely unexplainable, so all values are zero.  The
10 explainable flows in the third column are explainable so the answers are
fully explainable.  The last column shows how the average explainability is
diminished for the winning answers.  The MLP system has the lowest average
explainability, while the SVM system has the highest.

Figures~\ref{tab:mnist_conf_mlp} and \ref{tab:mnist_conf_svm} depict the
contrast between the confusion matrices on the unexplainable benchmark between
the MLP and SVM.  In the SVM, the highest value in a non-diagonal cell is
eleven.  Eleven occurs three times in the confusion matrix where some sevens are
erroneously predicted as a two and where the digits four and seven are
erroneously detected as a nine. The MLP confusion matrix shows in the first
column that many digits (especially the two, seven, eight, and nine) are
erroneously recognized as the digit zero.  The highest value in the non-diagonal
of the MLP confusion matrix is an eight, where fours are predicted as the digit
nine.

% \begin{figure}[H]
%     \centerline{\includegraphics[width=10cm]{./images/mnist_conf_matrix_mlp.png}}
%     \caption{Confusion matrix for the single unexplainable MLP on MNIST.}
%     \label{fig:mnist_conf_mlp}
% \end{figure}

\begin{table}[H]
    %\centering
    \caption{Confusion matrix for the single unexplainable MLP on MNIST.}
    \begin{center}
    \label{tab:mnist_conf_mlp}
    \renewcommand{\arraystretch}{1.3}
    \begin{tabular}{ll|c|c|c|c|c|c|c|c|c|c|}
        \multicolumn{2}{c}{}& \multicolumn{10}{c}{Predicted}\\
        & \multicolumn{1}{c}{} & \multicolumn{1}{c}{0} & \multicolumn{1}{c}{1} & \multicolumn{1}{c}{2}
        & \multicolumn{1}{c}{3} & \multicolumn{1}{c}{4} & \multicolumn{1}{c}{5} & \multicolumn{1}{c}{6}
        & \multicolumn{1}{c}{7} & \multicolumn{1}{c}{8} & \multicolumn{1}{c}{9} \\
        \cline{3-12}
        \multirow{10}{*}{{\rotatebox[origin=c]{90}{Actual}
        }} & 
        0 & $~973$ & 1 & 1 & 0 & 1 & 1 & 1 & 1 & 1 & 0 \\ \cline{3-12}
        &   1 & 4 & $1127$ & 0 & 1 & 0 & 0 & 2 & 0 & 1 & 0 \\ \cline{3-12}
        &   2 & 6 & 1 & $1016$ & 1 & 1 & 0 & 1 & 4 & 2 & 0 \\ \cline{3-12}
        &   3 & 4  & 0 & 2 & $~995$ & 0 & 3 & 0 & 1 & 3 & 2 \\ \cline{3-12}
        &   4 & 4 & 0 & 2 & 0 & $~962$ & 0 & 4 & 2 & 0 & 8 \\ \cline{3-12}
        &   5 & 3 & 0 & 0 & 9 & 1 & $~870$ & 3 & 1 & 3 & 2 \\ \cline{3-12}
        &   6 & 5 & 2 & 1 & 1 & 2 & 1 & $~945$ & 0 & 1 & 0 \\ \cline{3-12}
        &   7 & 7 & 0 & 6 & 1 & 1 & 0 & 0 & $1006$ & 2 & 5 \\ \cline{3-12}
        &   8 & 7 & 0 & 2 & 2 & 2 & 0 & 0 & 2 & $~955$ & 4 \\ \cline{3-12}
        &   9 & 7 & 2 & 0 & 3 & 6 & 3 & 1 & 6 & 2 & $~979$ \\ \cline{3-12}
    \end{tabular}
    \end{center}
\end{table}


% \begin{figure}[H]
%     \centerline{\includegraphics[width=10cm]{./images/mnist_conf_matrix_svm.png}}
%     \caption{Confusion matrix for the single unexplainable SVM on MNIST.}
%     \label{fig:mnist_conf_svm}
% \end{figure}


\begin{table}[H]
    %\centering
    \caption{Confusion matrix for the single unexplainable SVM on MNIST.}
    \begin{center}
    \label{tab:mnist_conf_svm}
    \renewcommand{\arraystretch}{1.3}
    \begin{tabular}{ll|c|c|c|c|c|c|c|c|c|c|}
        \multicolumn{2}{c}{}& \multicolumn{10}{c}{Predicted}\\
        & \multicolumn{1}{c}{} & \multicolumn{1}{c}{0} & \multicolumn{1}{c}{1} & \multicolumn{1}{c}{2}
        & \multicolumn{1}{c}{3} & \multicolumn{1}{c}{4} & \multicolumn{1}{c}{5} & \multicolumn{1}{c}{6}
        & \multicolumn{1}{c}{7} & \multicolumn{1}{c}{8} & \multicolumn{1}{c}{9} \\
        \cline{3-12}
        \multirow{10}{*}{{\rotatebox[origin=c]{90}{Actual}
        }} & 
        0 & $~973$ & 0 & 1 & 0 & 0 & 2 & 1 & 1 & 2 & 0 \\ \cline{3-12}
        &   1 & 0 & $1126$ & 3 & 1 & 0 & 1 & 1 & 1 & 2 & 0 \\ \cline{3-12}
        &   2 & 6 & 1 & $1006$ & 2 & 1 & 0 & 2 & 7 & 6 & 1 \\ \cline{3-12}
        &   3 & 0  & 0 & 2 & $~995$ & 0 & 2 & 0 & 5 & 5 & 1 \\ \cline{3-12}
        &   4 & 0 & 0 & 5 & 0 & $~961$ & 0 & 3 & 0 & 2 & 11 \\ \cline{3-12}
        &   5 & 2 & 0 & 0 & 9 & 0 & $~871$ & 4 & 1 & 4 & 1 \\ \cline{3-12}
        &   6 & 6 & 2 & 0 & 0 & 2 & 3 & $~944$ & 0 & 1 & 0 \\ \cline{3-12}
        &   7 & 0 & 6 & 11 & 1 & 1 & 0 & 0 & $996$ & 2 & 11 \\ \cline{3-12}
        &   8 & 3 & 0 & 2 & 6 & 3 & 2 & 2 & 3 & $~950$ & 3 \\ \cline{3-12}
        &   9 & 3 & 4 & 1 & 7 & 10 & 2 & 1 & 7 & 4 & $~970$ \\ \cline{3-12}
    \end{tabular}
    \end{center}
\end{table}

Table \ref{tab:mnist_conf_exp_cnn} illustrates the confusion matrix on the MNIST
test set for the highest performing explainable architecture using ten
explainable and one unexplainable Resnet50 flows. The confusion matrix shows
that thirteen times the digit four was erroneously predicted as a nine and in
ten instance a handwritten five was erroneously predicted as a three.

\begin{table}[H]
    %\centering
    \caption{Confusion matrix for an explainable Resnet50 architecture with ten explainable and one unexplainable flows on MNIST.}
    \begin{center}
    \label{tab:mnist_conf_exp_cnn}
    \renewcommand{\arraystretch}{1.3}
    \begin{tabular}{ll|c|c|c|c|c|c|c|c|c|c|}
        \multicolumn{2}{c}{}& \multicolumn{10}{c}{Predicted}\\
        & \multicolumn{1}{c}{} & \multicolumn{1}{c}{0} & \multicolumn{1}{c}{1} & \multicolumn{1}{c}{2}
        & \multicolumn{1}{c}{3} & \multicolumn{1}{c}{4} & \multicolumn{1}{c}{5} & \multicolumn{1}{c}{6}
        & \multicolumn{1}{c}{7} & \multicolumn{1}{c}{8} & \multicolumn{1}{c}{9} \\
        \cline{3-12}
        \multirow{10}{*}{{\rotatebox[origin=c]{90}{Actual}
        }} & 
        0 & $~976$ & 0 & 0 & 0 & 0 & 0 & 3 & 0 & 0 & 1 \\ \cline{3-12}
        &   1 & 0 & $1132$ & 0 & 0 & 0 & 0 & 2 & 1 & 0 & 0 \\ \cline{3-12}
        &   2 & 1 & 1 & $1024$ & 1 & 0 & 0 & 0 & 5 & 0 & 0 \\ \cline{3-12}
        &   3 & 0  & 2 & 1 & $1001$ & 0 & 0 & 0 & 5 & 1 & 0 \\ \cline{3-12}
        &   4 & 0 & 2 & 3 & 0 & $~960$ & 0 & 4 & 1 & 0 & 13 \\ \cline{3-12}
        &   5 & 3 & 0 & 0 & 10 & 0 & $~874$ & 2 & 1 & 2 & 0 \\ \cline{3-12}
        &   6 & 6 & 2 & 0 & 0 & 2 & 3 & $~946$ & 0 & 2 & 0 \\ \cline{3-12}
        &   7 & 0 & 8 & 3 & 0 & 0 & 0 & 0 & $1016$ & 1 & 0 \\ \cline{3-12}
        &   8 & 4 & 0 & 2 & 1 & 0 & 0 & 0 & 2 & $~960$ & 5 \\ \cline{3-12}
        &   9 & 3 & 1 & 0 & 1 & 7 & 0 & 0 & 3 & 0 & $~994$ \\ \cline{3-12}
    \end{tabular}
    \end{center}
\end{table}


% TODO: outline over allresults and present a table using different ANN
% architectures.

% TODO format as a table:
% \begin{itemize}
%     \item $0.979$ mlp 10 explainable 1 unexplainable
%     \item $0.962$ mlp 10 explainable
%     \item $0.983$ 1 mlp unexplainable
%     \item $0.973$ svm 10 explainable 1 unexplainable
%     \item $0.954$ svm 10 explainable
%     \item $0.9788$ 1 svm unexplainable rbf
%     \item $0.987$ cnn 10 explainable 1 unexplainable
%     \item $0.973$ cnn 10 explainable
%     \item $0.994$ 1 cnn unexplainable
%     \item $0.988$ resnet 10 explainable 1 unexplainable
%     \item $0.976$ resent 10 explainable
%     \item $0.989$ 1 resnet 50 unexplainable
% \end{itemize}

\subsection{MNIST Explainable Examples}
\label{sec:mnist_exp_prop_examples}

Detailed examples of explainable MNIST digit recognition, using the
property-based explainable method, follow in this section. Three interesting
MNIST digits are presented.  The property transform opinions, effectiveness,
confidence, and explainability will be reviewed in detail.  This is done by
stepping through the inference engine results for the digits, the procedures of
the probabilistic voting scheme, explainability metrics, and composing the
explainability rationale from the explainable property-based methodology.

The explainable architecture for these explainable samples is using eleven
pre-decision flows (PDF).  One flow is unexplainable using the untransformed
application input. Inference engines in the property-based explainable
architecture are multilayer perceptrons with three hidden layers of 128, 256,
and 128 neurons each. The neural networks are using the rectified linear unit
activation function and a stochastic gradient descent solver. These samples also
used the $E_{PARS}$ effectiveness metric in the property-based explainable
architecture's decision-making process.

When characterizing the confidence, $Conf(c)$, of a class ,$c$, for samples in
this section, the following confidence ranges apply:

\begin{itemize}
    \item High - $75.0 \leq Conf \leq 100.0$
    \item Medium - $25.0 \leq Conf < 75.0$
    \item Low - $0.0 \leq Conf < 25.0$
\end{itemize}

The confusion matrix from the MNIST test set for the explainable property-based architecture used
for digit examples is in Table~\ref{tab:mnist_example_conf_matrix}.  The confusion matrix
was stored in the knowledgebase and used to suggest how the system can fail for the examples.

\begin{table}[H]
    %\centering
    \caption{MLP explainable architecture confusion matrix for MNIST examples.}
    \begin{center}
    \label{tab:mnist_example_conf_matrix}
    \renewcommand{\arraystretch}{1.3}
    \begin{tabular}{ll|c|c|c|c|c|c|c|c|c|c|}
        \multicolumn{2}{c}{}& \multicolumn{10}{c}{Predicted}\\
        & \multicolumn{1}{c}{} & \multicolumn{1}{c}{0} & \multicolumn{1}{c}{1} & \multicolumn{1}{c}{2}
        & \multicolumn{1}{c}{3} & \multicolumn{1}{c}{4} & \multicolumn{1}{c}{5} & \multicolumn{1}{c}{6}
        & \multicolumn{1}{c}{7} & \multicolumn{1}{c}{8} & \multicolumn{1}{c}{9} \\
        \cline{3-12}
        \multirow{10}{*}{{\rotatebox[origin=c]{90}{Actual}
        }} & 
        0 &     972 &  0   &  1   &  0  &  0  &  0  &  7  &  0   &  0  &  0  \\ \cline{3-12}
        &   1 &  0  & 1131 &  1   &  1  &  0  &  0  &  2  &  0   &  0  &  0  \\ \cline{3-12}
        &   2 &  3  &  0   & 1015 &  0  &  0  &  0  &  4  &  8   &  2  &  0  \\ \cline{3-12}
        &   3 &  4  &  2   &  2   & 990 &  0  &  2  &  0  &  6   &  4  &  0  \\ \cline{3-12}
        &   4 &  4  &  2   &  2   &  0  & 953 &  0  &  4  &  4   &  2  & 11   \\ \cline{3-12}
        &   5 &  3  &  2   &  0   & 12  &  2  & 862 &  4  &  1   &  2  &  4  \\ \cline{3-12}
        &   6 &  2  &  4   &  0   &  0  &  2  &  2  & 947 &  0   &  1  &  0  \\ \cline{3-12}
        &   7 &  2  &  4   &  12  &  1  &  3  &  0  &  0  & 1000 &  2  &  4  \\ \cline{3-12}
        &   8 & 10  &  3   &  1   &  4  &  2  &  0  &  6  &  2   & 939 &  7  \\ \cline{3-12}
        &   9 &  7  &  3   &  0   &  3  &  9  &  1  &  1  &  3   &  1  & 981 \\ \cline{3-12}
    \end{tabular}
    \end{center}
\end{table}

% mnist test sample index 4 a digit 4
\begin{figure}[H]
    \centerline{\includegraphics[width=2cm, alt={An MNIST digit four}]{./images/mnist_samples/4_4_0.png}}
    \caption{MNIST example one, a digit four.}
    \label{fig:mnist_example1}
\end{figure}


The first MNIST example is shown in Figure~\ref{fig:mnist_example1}.  The digit is
index four (zero based indexing) in the MNIST test set.  The digit is labeled as
the number four. Table \ref{table:mnist_example1_ex} details the explainability
matrix which contains the inference engine results and explainability for each
flow.

Each row in the body of the explainability matrix represents a property
transform's flow. The first column lists the flow id as $F_j$ for flow $j$.
The property transforms associated with the flows correspond to the property
transforms in Table \ref{tab:transsample}.

The second column, labeled property, in the explainability matrix indicates the
property name associated with the flow.  The third column, labeled class vote,
indicates the opinion of the inference engine as the class it voted for. The
next three columns labeled with PDF effectiveness $E(j,c)$ indicate the
effectiveness of each flow $j$ for the class $c$ it voted for. The next three
columns labeled PDF explainability with $Ex(c)$ indicate the explainability
metric for each flow $j$ that voted for class $c$. These values are important in
calculating the explainability of the rationale for each class.

\begin{table}[h!]
    % \begin{minipage}{0.15\linewidth}
    %     \centering
    %     \includegraphics[width=2cm]{./images/mnist_samples/4_4_0.png}
    %     \captionof{figure}{A digit four example.}
    %     \label{fig:ex4}
    % \end{minipage}\hfill
    %\begin{minipage}{0.85\linewidth}
        \renewcommand{\arraystretch}{1.3}
        \captionof{table}{Explainability Matrix for MNIST Example one}\label{table:mnist_example1_ex}
        \begin{center}
        \resizebox{0.85\linewidth}{!}{%
        \begin{tabular}{| c | c | c | c | c | c || c | c | c |}
        \cline{4-9}
        \multicolumn{3}{c}{} & \multicolumn{3}{|c||}{PDF Effectiveness} & \multicolumn{3}{c|}{PDF Explainability} \\
        \hline
        $F_j$ & Property & Class Vote & $E(j,0)$ & $E(j,4)$ & $E(j,9)$ & $Ex(0)$ & $Ex(4)$ & $Ex(9)$ \\
        \hline
        \hline
        $F_1$ & Stroke & $4$ &  & $1.0$ &  &  & $1.0$ & \\ 
        \hline
        $F_2$ & Circle & $0$ & $0.039$ &  &  & $1.0$ &  & \\
        \hline
        $F_3$ & Crossing & $0$ & $0.018$ &  &  & $1.0$ &  & \\
        \hline
        $F_4$ & Ellipse & $0$ & $0.004$ &  &  & $1.0$ &  & \\
        \hline
        $F_5$ & Ell-Cir & $0$ & $0.069$ &  &  & $1.0$ &  & \\
        \hline
        $F_6$ & Endpoint & $4$ &  & $0.974$ &  &  & $1.0$ & \\
        \hline
        $F_7$ & Flood Fill & $0$ & $0.021$ &  &  & $1.0$ &  & \\
        \hline
        $F_8$ & Line & $9$ &  &  & $0.496$ &  &  & $1.0$ \\
        \hline
        $F_9$& Convex Hull & $4$ &  & $0.826$ &  &  & $1.0$ & \\
        \hline
        $F_{10}$& Corner & $4$ &  & $0.538$ &  &  & $1.0$ & \\
        \hline
        $F_{11}$& Unexplainable & $4$ &  & $1.0$ &  &  & $0.0$ & \\
        \hline
        \hline
        \multicolumn{3}{|c|}{Weights - $WE(c)$ / $\sum E(j,c)X_j$} & $0.151$ & $4.337$ & $0.496$ & $0.151$ & $3.337$ & $0.496$ \\
        \cline{0-8}
        \multicolumn{3}{|c|}{Confidence / Explainability} & $3.03\%$ & $87.0\%$ & $9.96\%$ & $100\%$ & $76.9\%$ & $100\%$ \\
        \cline{0-8}
        \end{tabular}
        }
        \end{center}
    %\end{minipage}
\end{table}

Observing the class votes in Table \ref{table:mnist_example1_ex} for example
one, five of the PDFs suggested the digit four while five other PDFs suggested
the digit zero.  The effectiveness of the flows that suggested the four were
much higher with an effectiveness range from $0.538$ to $1.0$.  Effectiveness of
the PDFs suggesting the digit zero had a comparatively much lower range from
$0.004$ to $0.069$.  The effectiveness columns are summed to provide the
weighted effectiveness, $WE_c$, for each class, $c$, that was voted for. The
class indicating the digit four has a much higher weighted effectiveness, at
$4.337$, while the next highest is for the digit nine at $0.496$. Confidence
values for each class, $c$, that was voted for are listed in the last row of the
table under the Effectiveness columns. Observe that the confidence for the class
four is at $87.0\%$.

The explainability columns in the explainability matrix indicate the
explainability metric, $Ex(c)$, of the flows that voted for class, $c$.
In these examples, the explainability metric will be $1.0$ for all PDFs but the
unexplainable PDF, where the metric will be $0.0$ due to the lack of
explainability.  In calculating the explainability for each class $c$, the
product of effectiveness and the explainbility metric is summed for each class
resulting in an explainbility that is weighted by the effectiveness of the PDFs.
This results in the explainability resulting in $1.0$ for classes that received
only votes for explainable properties.  Classes receiving unexplainable votes
will have the explainability diminished relative to the effectiveness of the
unexplainable flow for that class.  This is depicted in the explainability,
$Ex(c)$ from \eqref{eq:explainability}, for the class $c$ as the digit four
across the flows $j$ that suggested the digit four in
\eqref{eq:explainability_mnist_example1}.

\begin{equation}\label{eq:explainability_mnist_example1}
    Ex(4)=\frac{\sum{E(j,4)X_j}}{\sum{E(j,4)}} = \frac{3.337}{4.337}=0.769
\end{equation}

\begin{figure}[H]
    \renewcommand{\arraystretch}{1.3}
    \captionof{table}{Explanation for MNIST Example one} \label{table:mnist_example1_explanation}
    %\centering
    \begin{center}
    %\resizebox{\columnwidth}{!}{%
    \begin{tabular}{| m{0.06\linewidth} | m{0.14\linewidth} | m{0.17\linewidth} | m{0.55\linewidth} |}
    \hline
     Class & Confidence & Explainability & Explainable Description \\
    \hline \hline
    $4$ & $87.0\%$ & $76.9\%$ & Confidence is high for interpreting this character as a four due to the stroke, endpoint, convex hull, and corner properties. \\ 
    \hline
    $9$ & $9.96\%$ & $100\%$ & Confidence is low for interpreting this character as a nine due to the line property. \\
    \hline
    $0$ & $3.03\%$ & $100\%$ & Confidence is low for interpreting this character as a zero due to the ellipse-circle, circle, flood fill, crossing, and ellipse properties. \\
    \hline
    \end{tabular}
    %}
    \end{center}
\end{figure}

Table \ref{table:mnist_example1_explanation} presents the results from the XAI
component of the architecture. The decisions are presented in descending order
based on confidence.  Note that since the unexplainable flow voted for the digit
four the explainability for the winning solution is diminished at $76.9\%$.
Observe that the digit four won with $87.0\%$ confidence and an explainability
of $76.9\%$.  The rationale provided by XAI for the winning digit indicates:

\begin{quote}
    \textit{Confidence is high for interpreting this character as a four due to the stroke, endpoint, convex hull, and corner properties.}
\end{quote}

Alternatives (the digits nine and zero) that were suggested by other properties
are also presented by XAI with supporting confidence ($9.96\%$ and $3.03\%$),
explainability, and rationale referencing explainable properties that suggested
the digits.

In addition to the predictions and rationale, the explainable system references
the confusion matrix in the knowledgebase and suggests that the false discovery
rate when predicting a four is $1.9\%$.  The system also suggests that the most
confused digits when predicting a four are a nine at $0.9\%$ and a seven at
$0.3\%$.

% mnist test sample index 1 a digit 2
\begin{figure}[H]
    \centerline{\includegraphics[width=2cm, alt={An MNIST digit two}]{./images/mnist_samples/1_2_0.png}}
    \caption{MNIST example two, a digit two.}
    \label{fig:mnist_example2}
\end{figure}

Example two is shown in Figure~\ref{fig:mnist_example2} which is index one (zero
based indexing) in the MNIST test set. The digit is labeled as a two.

Table \ref{table:mnist_example2_ex} shows the explainability matrix for example
two. Six of the flows, including the unexplainable flow, suggested digit two
with effectiveness ranging from $0.564$ to $1.0$. Four flows suggested the digit
zero with very low effectiveness from $0.004$ to $0.039$. One flow, the ellipse
circle, suggested the digit eight with effectiveness $0.288$.  It is clear from
these effectiveness metrics that the two is preferred by the system as the most
flows suggested the digit two and those flows had comparatively high
effectiveness metrics.

The resulting weighted effectiveness in Table \ref{table:mnist_example2_ex}
heavily favors the digit two, with confidence $93.1\%$, over the eight, with
$5.37\%$ confidence, and the zero, with $1.53\%$ confidence.

Because the unexplainable flow suggested the two, the explainability of the
recommended answer is impacted.  We see the effectiveness of the unexplainable
flow, $F_{11}$, was $1.0$.  This results in the explainability, $Ex(c)$ from
\eqref{eq:explainability}, with the class $c=2$ resulting in $80.0\%$ as shown
in \eqref{eq:explainability_mnist_example2}.

\begin{equation}\label{eq:explainability_mnist_example2}
    Ex(2)=\frac{\sum{E(j,2)X_j}}{\sum{E(j,2)}} = \frac{3.99}{4.99}=0.80
\end{equation}

\begin{table}[H]
    % \begin{minipage}{0.15\linewidth}
    %     \centering
    %     \includegraphics[width=2cm]{./images/mnist_samples/4_4_0.png}
    %     \captionof{figure}{A digit four example.}
    %     \label{fig:ex4}
    % \end{minipage}\hfill
    %\begin{minipage}{0.85\linewidth}
        \renewcommand{\arraystretch}{1.3}
        \captionof{table}{Explainability Matrix for MNIST Example two}\label{table:mnist_example2_ex}
        %\centering
        \begin{center}
        \resizebox{0.85\linewidth}{!}{%
        \begin{tabular}{| c | c | c | c | c | c || c | c | c |}
        \cline{4-9}
        \multicolumn{3}{c}{} & \multicolumn{3}{|c||}{PDF Effectiveness} & \multicolumn{3}{c|}{PDF Explainability} \\
        \hline
        $F_j$ & Property & Class Vote & $E(j,0)$ & $E(j,2)$ & $E(j,8)$ & $Ex(0)$ & $Ex(2)$ & $Ex(8)$ \\
        \hline
        \hline
        $F_1$ & Stroke & $2$ &  & $1.0$ &  &  & $1.0$ & \\ 
        \hline
        $F_2$ & Circle & $0$ & $0.039$ &  &  & $1.0$ &  & \\
        \hline
        $F_3$ & Crossing & $0$ & $0.018$ &  &  & $1.0$ &  & \\
        \hline
        $F_4$ & Ellipse & $0$ & $0.004$ &  &  & $1.0$ &  & \\
        \hline
        $F_5$ & Ell-Cir & $8$ &  &  & $0.288$ &  &  & $1.0$ \\
        \hline
        $F_6$ & Endpoint & $2$ &  & $0.901$ &  &  & $1.0$ & \\
        \hline
        $F_7$ & Flood Fill & $0$ & $0.021$ &  &  & $1.0$ &  & \\
        \hline
        $F_8$ & Line & $2$ &  & $0.564$ &  &  & $1.0$ & \\
        \hline
        $F_9$& Convex Hull & $2$ &  & $0.796$ &  &  & $1.0$ & \\
        \hline
        $F_{10}$& Corner & $2$ &  & $0.729$ &  &  & $1.0$ & \\
        \hline
        $F_{11}$& Unexplainable & $2$ &  & $1.0$ &  &  & $0.0$ & \\
        \hline
        \hline
        \multicolumn{3}{|c|}{Weights - $WE(c)$ / $\sum E(j,c)X_j$} & $0.082$ & $4.990$ & $0.288$ & $0.082$ & $3.990$ & $0.288$ \\
        \cline{0-8}
        \multicolumn{3}{|c|}{Confidence / Explainability} & $1.53\%$ & $93.1\%$ & $5.37\%$ & $100\%$ & $80.0\%$ & $100\%$ \\
        \cline{0-8}
        \end{tabular}
        }
        \end{center}
    %\end{minipage}
\end{table}

Table \ref{table:mnist_example2_explanation} depicts the explainable results
from XAI. The digit two wins with the alternatives presented with low confidence
as eight and zero.  The rationale provided for the winning selection indicates:

\begin{quote}
    \textit{Confidence is high for interpreting this character as a two due to the stroke, endpoint, convex hull, corner, and line properties.}
\end{quote}

\begin{table}[H]
    \renewcommand{\arraystretch}{1.3}
    \captionof{table}{Explanation for MNIST Example two} \label{table:mnist_example2_explanation}
    %\centering
    \begin{center}
    %\resizebox{\columnwidth}{!}{%
    \begin{tabular}{| m{0.06\linewidth} | m{0.14\linewidth} | m{0.17\linewidth} | m{0.55\linewidth} |}
    \hline
     Class & Confidence & Explainability & Explainable Description \\
    \hline \hline
    $2$ & $93.1\%$ & $80.0\%$ & Confidence is high for interpreting this character as a two due to the stroke, endpoint, convex hull, corner, and line properties. \\ 
    \hline
    $8$ & $5.37\%$ & $100\%$ & Confidence is low for interpreting this character as an eight due to the ellipse-circle property. \\
    \hline
    $0$ & $1.53\%$ & $100\%$ & Confidence is low for interpreting this character as a zero due to the circle, flood fill, crossing, and ellipse properties. \\
    \hline
    \end{tabular}
    %}
    \end{center}
\end{table}

The explainable system references the confusion matrix in the knowledgebase and
suggests that the false discovery rate when predicting a two is $1.8\%$. The
system elaborates that the most confused digit when predicting a two is a seven
at $1.2\%$.

Example three is shown in Figure~\ref{fig:mnist_example3} which is index seven (zero
based indexing) in the MNIST test set. The digit is labeled as a nine.

% mnist test sample index 7 a digit 9
\begin{figure}[H]
    \centerline{\includegraphics[width=2cm, alt={An MNIST digit nine}]{./images/mnist_samples/7_9_0.png}}
    \caption{MNIST example three, a digit nine.}
    \label{fig:mnist_example3}
\end{figure}
%\vspace{\floatsep}

Table \ref{table:mnist_example3_ex} presents the explainability matrix for
example three.  In this example, five classes were suggested by the inference
engines, zero, three, four, seven, and nine.  Including the unexplainable flow,
five flows suggested the digit nine, three flows suggested the digit zero, and
one flow each suggested the digits three, four, and seven.

\begin{table}[H]
    % \begin{minipage}{0.15\linewidth}
    %     \centering
    %     \includegraphics[width=2cm]{./images/mnist_samples/4_4_0.png}
    %     \captionof{figure}{A digit four example.}
    %     \label{fig:ex4}
    % \end{minipage}\hfill
    %\begin{minipage}{0.85\linewidth}
        \renewcommand{\arraystretch}{1.3}
        \captionof{table}{Explainability Matrix for MNIST Example three}\label{table:mnist_example3_ex}
        %\centering
        \begin{center}
        \resizebox{\linewidth}{!}{%
        \begin{tabular}{| c | c | c | c | c | c | c | c || c | c | c | c | c |}
        \cline{4-13}
        \multicolumn{3}{c}{} & \multicolumn{5}{|c||}{PDF Effectiveness} & \multicolumn{5}{c|}{PDF Explainability} \\
        \hline
        $F_j$ & Property & Class Vote & $E(j,0)$ & $E(j,3)$ & $E(j,4)$ & $E(j,7)$ & $E(j,9)$ & $Ex(0)$ & $Ex(3)$ & $Ex(4)$ & $Ex(7)$ & $Ex(9)$ \\
        \hline
        \hline
        $F_1$ & Stroke & $9$ &  &  &  &  & $1.0$ &  &  &  &  & $1.0$\\ 
        \hline
        $F_2$ & Circle & $0$ & $0.039$ &  &  &  &  & $1.0$ &  &  &  & \\
        \hline
        $F_3$ & Crossing & $3$ &  & $0.220$ &  &  &  &  & $1.0$ &  &  & \\
        \hline
        $F_4$ & Ellipse & $0$ & $0.004$ &  &  &  &  & $1.0$ &  &  &  & \\
        \hline
        $F_5$ & Ell-Cir & $0$ & $0.069$ &  &  &  &  & $1.0$ &  &  &  & \\
        \hline
        $F_6$ & Endpoint & $9$ &  &  &  &  & $0.920$ &  &  &  &  & $1.0$ \\
        \hline
        $F_7$ & Flood Fill & $9$ &  &  &  &  & $0.789$ &  &  &  &  & $1.0$ \\
        \hline
        $F_8$ & Line & $4$ &  &  & $0.562$ &  &  &  &  & $1.0$ &  & \\
        \hline
        $F_9$& Convex Hull & $9$ &  &  &  &  & $0.739$ &  &  &  &  & $1.0$ \\
        \hline
        $F_{10}$& Corner & $7$ &  &  &  & $0.697$ &  &  &  &  & $1.0$ &  \\
        \hline
        $F_{11}$& Unexplainable & $9$ &  &  &  &  & $1.0$ & & & & & $0.0$ \\
        \hline
        \hline
        \multicolumn{3}{|c|}{Weights - $WE(c)$ / $\sum E(j,c)X_j$} & $0.112$ & $0.221$ & $0.562$ & $0.697$ & $4.449$ & $0.112$ & $0.221$ & $0.562$ & $0.697$ & $3.449$ \\
        \cline{0-12}
        \multicolumn{3}{|c|}{Confidence / Explainability} & $1.85\%$ & $3.65\%$ & $9.30\%$ & $11.5\%$ & $73.7\%$ & $100\%$ & $100\%$ & $100\%$ & $100\%$ & $77.5\%$ \\
        \cline{0-12}
        \end{tabular}
        }
        \end{center}
    %\end{minipage}
\end{table}

The effectiveness associated with the nine results in the nine winning with
$73.7\%$ confidence.  The unexplainable flow also suggested the digit nine. As a
result, the explainability metric for nine is $77.5\%$ as shown in
\eqref{eq:explainability_mnist_example3}.  In terms of confidence, the next
closest alternative to the nine is the digit seven with $11.5\%$ confidence.

\begin{equation}\label{eq:explainability_mnist_example3}
    Ex(9)=\frac{\sum{E(j,9)X_j}}{\sum{E(j,9)}} = \frac{3.449}{4.449}=0.775
\end{equation}


\begin{table}[H]
    \renewcommand{\arraystretch}{1.3}
    \captionof{table}{Explanation for MNIST Example three} \label{table:mnist_example3_explanation}
    %\centering
    \begin{center}
    %\resizebox{\columnwidth}{!}{%
    \begin{tabular}{| m{0.06\linewidth} | m{0.14\linewidth} | m{0.17\linewidth} | m{0.55\linewidth} |}
    \hline
     Class & Confidence & Explainability & Explainable Description \\
    \hline \hline
    $9$ & $73.7\%$ & $77.5\%$ & Confidence is medium for interpreting this character as a nine due to the stroke, endpoint, flood fill, and convex hull properties. \\ 
    \hline
    $7$ & $11.5\%$ & $100\%$ & Confidence is low for interpreting this character as a seven due to the corner property. \\
    \hline
    $4$ & $9.30\%$ & $100\%$ & Confidence is low for interpreting this character as a four due to the line property. \\
    \hline
    $3$ & $3.65\%$ & $100\%$ & Confidence is low for interpreting this character as a three due to the crossing property. \\
    \hline
    $0$ & $1.85\%$ & $100\%$ & Confidence is low for interpreting this character as a zero due to the ellipse-circle, circle, and ellipse properties. \\
    \hline
    \end{tabular}
    %}
    \end{center}
\end{table}

Table \ref{table:mnist_example3_explanation} presents the explainable description from the
XAI block.  The rationale provided by XAI for the winning digit nine follows:

\begin{quote}
    \textit{Confidence is medium for interpreting this character as a nine due to the stroke, endpoint, fill, and convex hull properties.}
\end{quote}

The explainable system references the confusion matrix in the knowledgebase and
suggests that the false discovery rate when predicting a nine is $2.6\%$. The
system further suggests that the most confused digits when predicting a nine are
a four at $1.1\%$ and an eight at $0.7\%$.

Accuracy achieved with an explainable property-based system previously with
MNIST was $91.9\%$ with $100\%$ explainability\cite{whitten21}.  In that case
nine explainable properties, multi-layer perceptron inference engines, and the
recall metric were used for gauging effectiveness. Utilizing an unexplainable
component (consisting of the original untransformed image), an additional convex
hull transform, and a better metric for effectiveness resulted in an improvement
to $97.9\%$ accuracy from the MLP and $98.8\%$ for Resnet50. The resulting
average explainability metric, $Ex(c)$, for the winning classes, $c$, was
reduced to $67.2$\% for MLP and $69.9\%$ for Resnet50.

\subsection{EMNIST Results}



%Results for addition of the convex hull are 97.1\% accuracy.  TODO add more details
%here.

%Results for adding the two hidden layer CNN are 96.0\% versus 91.9, a 4\%
%increase on the MNIST dataset from \cite{whitten21}.  TODO: add more here.

Because of considerable ambiguity between classes in the balanced EMNIST
dataset, it was split into three groups: digits, uppercase characters, and
lowercase characters. Table~\ref{tab_unexplainable_accuracy_results} shows the
benchmark accuracy results of a single unexplainable ML model using the
indicated model architectures on the full balanced EMNIST dataset and the three
distinct splits. Accuracy is much higher for the splits (digits, uppercase, and
lowercase) than the full dataset. On the full balanced dataset, with 46 classes,
Resnet50 slightly outperformed the CNN. The two-layer CNN can be observed as
having the maximum accuracy results in digits, uppercase, and lowercase
characters.

% \begin{table}[H]
%     \renewcommand{\arraystretch}{1.3}
%     \centering
%     \caption{Unexplainable accuracy results on various balanced EMNIST data sets with differing ML models}
%     %{\columnwidth}{!}{%
%     \begin{tabular}{|l|c|c|c|c|c|c|c|c|}
%         \cline{2-9}
%         \multicolumn{1}{c}{} & \multicolumn{8}{|c|}{Accuracy (\%)} \\
%         \cline{2-9}
%         \multicolumn{1}{c}{} & \multicolumn{2}{|c|}{Full EMNIST} & \multicolumn{2}{|c|}{EMNIST Digits} & \multicolumn{2}{|c|}{EMNIST Caps} & \multicolumn{2}{|c|}{EMNIST Lower} \\
%         \hline
%         Model Type & Train & Test & Train & Test & Train & Test & Train & Test \\
%         \hline
%         \hline
%         MLP & 82.59 & 78.68 & 99.82 & 97.32 & 95.88 & 91.13 & 97.91 & 89.47 \\
%         \hline
%         CNN & 89.14 & 88.66 & 99.44 & 99.05 & 95.87 & 95.36 & 94.84 & 93.41 \\
%         \hline
%         Resnet50 & 88.69 & 88.68 & 99.44 & 98.90 & 95.26 & 94.89 & 97.07 & 92.68 \\
%         \hline
%     \end{tabular}
%     %}
%     \label{tab_unexplainable_accuracy_results}
% \end{table}

\begin{table}[H]
    \renewcommand{\arraystretch}{1.3}
    %\centering
    \caption{Benchmark unexplainable accuracy results on the full and various splits of the balanced EMNIST data sets with differing ML models.}
    \begin{center}
    \resizebox{\linewidth}{!}{%
    \begin{tabular}{|l|c|c|c|c|}
        \cline{2-5}
        \multicolumn{1}{c}{} & \multicolumn{4}{|c|}{Accuracy (\%)} \\
        \hline
        \multicolumn{1}{|p{3cm}|}{\centering Model\\Type} & \multicolumn{1}{|p{3cm}|}{\centering Full\\EMNIST} & \multicolumn{1}{|p{3cm}|}{\centering EMNIST\\Digits} & \multicolumn{1}{|p{3cm}|}{\centering EMNIST\\Uppercase} & \multicolumn{1}{|p{3cm}|}{\centering EMNIST\\Lowercase} \\
        \hline
        \hline
        MLP & $74.3$ & $97.8$ & $88.7$ & $89.4$ \\
        \hline
        SVM & $84.3$ & $97.7$ & $91.8$ & $90.8$ \\
        \hline
        CNN & $88.7$ & $98.8$ & $95.4$ & $93.5$ \\
        \hline
        Resnet50 & $89.0$ & $98.3$ & $94.9$ & $93.4$ \\
        \hline
    \end{tabular}
    }
    \end{center}
    \label{tab_unexplainable_accuracy_results}
\end{table}

Table \ref{tab_explainable_accuracy_results} shows the explainable accuracy
results on the unpruned data. Compared to Table
\ref{tab_unexplainable_accuracy_results} the results are considerably lower than
the benchmark unexplainable results in Table
\ref{tab_unexplainable_accuracy_results}.  As expected, there appears to be a
cost for explainability.  That cost is decreased accuracy. In the explainable
results, the Resnet50 model performs best on full EMNIST, but CNN models
performed slightly better on the splits.

\begin{table}[H]
    \renewcommand{\arraystretch}{1.3}
    %\centering
    \caption{Explainable accuracy results on the full and various splits of the balanced EMNIST data sets with differing ML models using ten explainable flows.}
    \begin{center}
    \resizebox{\linewidth}{!}{%
    \begin{tabular}{|l|c|c|c|c|}
        \cline{2-5}
        \multicolumn{1}{c}{} & \multicolumn{4}{|c|}{Accuracy (\%)} \\
        \hline
        \multicolumn{1}{|p{3cm}|}{\centering Model\\Type} & \multicolumn{1}{|p{3cm}|}{\centering Full \\ EMNIST} & \multicolumn{1}{|p{3cm}|}{\centering EMNIST\\Digits} & \multicolumn{1}{|p{3cm}|}{\centering EMNIST\\Uppercase} & \multicolumn{1}{|p{3cm}|}{\centering EMNIST\\Lowercase} \\
        \hline
        \hline
        MLP  & $51.0$  & $87.1$  & $67.9$  & $68.3$ \\
        \hline
        SVM & $75.5$ & $93.9$ & $84.1$ & $84.2$ \\
        \hline
        CNN &  $79.5$ &  $96.2$ &  $89.2$ &  $89.7$ \\
        \hline
        Resnet50  & $82.6$  & $86.9$ &  $87.7$ & $85.6$ \\
        \hline
    \end{tabular}
    }
    \end{center}
    \label{tab_explainable_accuracy_results}
\end{table}


Table \ref{tab_exp_unexp_accuracy_results_no_thresh} shows adding an unexplainable component to the
PBE architecture. The accuracy results are improved significantly where the accuracy approaches
the benchmark unexplainable accuracy in Table \ref{tab_unexplainable_accuracy_results}.


\begin{table}[H]
    \renewcommand{\arraystretch}{1.3}
    %\centering
    \caption{Explainable accuracy results on the full and various splits of the balanced EMNIST data sets with differing ML models using ten explainable and one unexplainable flow.}
    \begin{center}
    \resizebox{\linewidth}{!}{%
    \begin{tabular}{|l|c|c|c|c|}
        \cline{2-5}
        \multicolumn{1}{c}{} & \multicolumn{4}{|c|}{Accuracy (\%)} \\
        \hline
        \multicolumn{1}{|p{3cm}|}{\centering Model\\Type} & \multicolumn{1}{|p{3cm}|}{\centering Full \\ EMNIST} & \multicolumn{1}{|p{3cm}|}{\centering EMNIST\\Digits} & \multicolumn{1}{|p{3cm}|}{\centering EMNIST\\Uppercase} & \multicolumn{1}{|p{3cm}|}{\centering EMNIST\\Lowercase} \\
        \hline
        \hline
        MLP  & $73.5$  & $94.5$  & $85.6$  & $84.9$ \\
        \hline
        SVM & $80.2$ & $96.3$ & $88.6$ & $88.2$ \\
        \hline
        CNN &  $85.6$ &  $97.8$ &  $93.0$ &  $90.8$ \\
        \hline
        Resnet50  & $83.8$  & $96.2$ &  $90.8$ & $92.1$ \\
        \hline
    \end{tabular}
    }
    \end{center}
    \label{tab_exp_unexp_accuracy_results_no_thresh}
\end{table}

Table \ref{tab_exp_unexp_explainability_results_no_thresh} illustrates the
impact on the average explainablility of the winning choice with an explainable
component in the PBE architecture. The multi-layer perceptron has an especially
low explainability, below 50 indicating much of the contribution for decisions
is due to the unexplainable flow rather than those associated with a property.

\begin{table}[H]
    \renewcommand{\arraystretch}{1.3}
    %\centering
    \caption{Average explainability results for the winning vote on the full and various splits of the balanced EMNIST data sets with differing ML models using ten explainable and one unexplainable flow.}
    \begin{center}
    \resizebox{\linewidth}{!}{%
    \begin{tabular}{|l|c|c|c|c|}
        \cline{2-5}
        \multicolumn{1}{c}{} & \multicolumn{4}{|c|}{Average Explainability (\%)} \\
        \hline
        \multicolumn{1}{|p{3cm}|}{\centering Model\\Type} & \multicolumn{1}{|p{3cm}|}{\centering Full \\ EMNIST} & \multicolumn{1}{|p{3cm}|}{\centering EMNIST\\Digits} & \multicolumn{1}{|p{3cm}|}{\centering EMNIST\\Uppercase} & \multicolumn{1}{|p{3cm}|}{\centering EMNIST\\Lowercase} \\
        \hline
        \hline
        MLP  & $48.2$  & $71.2$  & $57.8$  & $70.3$ \\
        \hline
        SVM & $69.6$ & $77.4$ & $71.3$ & $76.3$ \\
        \hline
        CNN &  $64.0$ &  $76.5$ &  $68.7$ &  $73.4$ \\
        \hline
        Resnet50  & $66.1$  & $63.6$ &  $77.7$ & $73.7$ \\
        \hline
    \end{tabular}
    }
    \end{center}
    \label{tab_exp_unexp_explainability_results_no_thresh}
\end{table}


% \begin{table}[H]
%     \renewcommand{\arraystretch}{1.3}
%     \centering
%     \caption{Explainable accuracy results on the full and various splits of the balanced EMNIST data sets with differing ML models using ten explainable and one unexplainable flow with a $0.05$ threshold.}
%     \resizebox{\linewidth}{!}{%
%     \begin{tabular}{|l|c|c|c|c|}
%         \cline{2-5}
%         \multicolumn{1}{c}{} & \multicolumn{4}{|c|}{Accuracy (\%)} \\
%         \hline
%         \multicolumn{1}{|p{3cm}|}{\centering Model\\Type} & \multicolumn{1}{|p{3cm}|}{\centering Full \\ EMNIST} & \multicolumn{1}{|p{3cm}|}{\centering EMNIST\\Digits} & \multicolumn{1}{|p{3cm}|}{\centering EMNIST\\Uppercase} & \multicolumn{1}{|p{3cm}|}{\centering EMNIST\\Lowercase} \\
%         \hline
%         \hline
%         MLP  & $73.6$  & $94.6$  & $85.6$  & $86.6$ \\
%         \hline
%         SVM & $80.3$ & $96.3$ & $88.7$ & $88.8$ \\
%         \hline
%         CNN &  $85.6$ &  $97.7$ &  $93.0$ &  $93.5$ \\
%         \hline
%         Resnet50  & $83.8$  & $96.2$ &  $90.8$ & $93.6$ \\
%         \hline
%     \end{tabular}
%     }
%     \label{tab_exp_unexp_accuracy_results_thresh}
% \end{table}

% \begin{table}[H]
%     \renewcommand{\arraystretch}{1.3}
%     \centering
%     \caption{Average explainability results for the winning vote on the full and various splits of the balanced EMNIST data sets with differing ML models using ten explainable and one unexplainable flow with a $0.05$ threshold.}
%     \resizebox{\linewidth}{!}{%
%     \begin{tabular}{|l|c|c|c|c|}
%         \cline{2-5}
%         \multicolumn{1}{c}{} & \multicolumn{4}{|c|}{Average Explainability (\%)} \\
%         \hline
%         \multicolumn{1}{|p{3cm}|}{\centering Model\\Type} & \multicolumn{1}{|p{3cm}|}{\centering Full \\ EMNIST} & \multicolumn{1}{|p{3cm}|}{\centering EMNIST\\Digits} & \multicolumn{1}{|p{3cm}|}{\centering EMNIST\\Uppercase} & \multicolumn{1}{|p{3cm}|}{\centering EMNIST\\Lowercase} \\
%         \hline
%         \hline
%         MLP  & $46.9$  & $71.1$  & $57.7$  & $56.8$ \\
%         \hline
%         SVM & $69.5$ & $77.3$ & $71.3$ & $71.3$ \\
%         \hline
%         CNN &  $63.9$ &  $76.5$ &  $68.7$ &  $68.8$ \\
%         \hline
%         Resnet50  & $66.1$  & $63.4$ &  $77.7$ & $67.9$ \\
%         \hline
%     \end{tabular}
%     }
%     \label{tab_exp_unexp_explainability_results_thresh}
% \end{table}


\subsection{EMNIST Explainable Examples}

This section depicts the explainability results using examples from the EMNIST
balanced test set using only explainable properties.  EMNIST example one, which
is labeled as an uppercase Q, is shown in Figure~\ref{fig:ex1}.  The sample is
index 71 in the balanced EMNIST test set.  Table \ref{table:emnist_example1_ex}
shows that votes for classes Q, G, and W were made by various properties.  Nine
of the properties contributed a vote for Q and one property each voted on the
other classes.  Summing the effectiveness weights for the classes, Q wins with a
confidence of 92.4\%.

\begin{figure}[H]
    \centering
    \includegraphics[width=2cm, alt={An EMNIST character labeled as an uppercase Q}]{./images/examples/test-Q-3.png}
    \caption{EMNIST example one, a test sample labeled Q.}
    \label{fig:ex1}
\end{figure}

\begin{table}[H]
    \renewcommand{\arraystretch}{1.3}
    \caption{Explainability matrix for EMNIST example one.}\label{table:emnist_example1_ex}
    %\centering
    \begin{center}
    \resizebox{\linewidth}{!}{%
    \begin{tabular}{| c | c | c | c | c | c || c | c | c |}
    \cline{4-9}
    \multicolumn{3}{c}{} & \multicolumn{3}{|c||}{Effectiveness} & \multicolumn{3}{c|}{Explainability} \\
    \hline
    $F_j$ & Property & Vote & $E(j,G)$ & $E(j,Q)$ & $Ex(W)$ & $Ex(G)$ & $Ex(Q)$ & $Ex(W)$ \\
    \hline
    \hline
    $F_0$ & Stroke & Q &  & $0.973$ &  &  & 1.0 &  \\ 
    \hline
    $F_1$ & Circle & Q &  & $0.128$ &  &  & $1.0$ &    \\
    \hline
    $F_2$ & Crossing & Q &  & $0.327$ &  &  & $1.0$ &   \\
    \hline
    $F_3$ & Ellipse & G & $0.168$ &  &  & $1.0$ &  &   \\
    \hline
    $F_4$ & Ell-Cir & W &  &  & $0.250$ &  &  & $1.0$  \\
    \hline
    $F_5$ & Endpoint & Q &  & $0.596$ &  &  & $1.0$ &   \\
    \hline
    $F_6$ & Flood Fill & Q &  & $0.888$ &  &  &$1.0$ &   \\
    \hline
    $F_7$ & Line & Q &  & $0.271$ &  &  & $1.0$ &    \\
    \hline
    $F_8$ & Convex Hull & Q &  & $0.670$ &  &  & $1.0$ &   \\
    \hline
    $F_9$ & Corner & Q &  & $0.257$ &  &  &  $1.0$ &   \\
    \hline
    $F_{10}$ & Unexplainable & Q &  & $0.992$ &  &  & $0.0$ &  \\
    \hline
    \hline
    \multicolumn{3}{|c|}{Weights - $WE(c)$ / $\sum E(j,c)X_j$} & $0.168$ & $5.102$ & $0.250$ & $0.167$ & $4.111$ & $0.250$ \\
    \cline{0-8}
    \multicolumn{3}{|c|}{Confidence/Explainability} & $3.03\%$ & $92.4\%$ & $4.52\%$ & $100\%$ & $80.6\%$ & $100\%$ \\
    \cline{0-8}
    \end{tabular}%
    }
    \end{center}
\end{table}

Explainable rationale provided by XAI is depicted in
Table~\ref{table:emnist_example1_explanation} for the three classes that received
votes from the NNs.

\begin{table}[H]
    \renewcommand{\arraystretch}{1.3}
    \captionof{table}{Explanation for EMNIST Example one} \label{table:emnist_example1_explanation}
    %\centering
    \begin{center}
    %\resizebox{\columnwidth}{!}{%
    \begin{tabular}{| m{0.06\linewidth} | m{0.14\linewidth} | m{0.17\linewidth} | m{0.55\linewidth} |}
    \hline
     Class & Confidence & Explainability & Explainable Description \\
    \hline \hline
    Q & $92.4\%$ & $80.6\%$ & Confidence is high for interpreting this character as a Q due to the stroke, enclosed region, convex hull, endpoint, crossing, line, corner, and circle properties. \\ 
    \hline
    W & $4.52\%$ & $100\%$ & Confidence is low for interpreting this character as a W due to the ellipse-circle property. \\
    \hline
    G & $3.03\%$ & $100\%$ & Confidence is low for interpreting this character as a G due to the ellipse property. \\
    \hline
    \end{tabular}
    %}
    \end{center}
\end{table}

Figure \ref{fig:ex2} shows example two, index 30 from the EMNIST balanced test
set, which is labeled as an uppercase C. Votes by the properties in the
explainability matrix shown in Table \ref{table:emnist_example2_ex} that eight
properties, including the unexplainable property, infer an uppercase C. The
overall weighted effectiveness for the class C is $3.801$ giving it a confidence
of $93.6\%$. The uppercase X receives a vote from the enclosed region property.
Uppercase T receives a vote from the ellipse property. An uppercase U receives a
vote from the line property.  Because effectiveness of the solitary opinions are
very low in comparison, C wins with $93.6\%$ confidence.

\begin{figure}[H]
    \centering
    \includegraphics[width=2cm, alt={An EMNIST character labeled as an uppercase C}]{./images/examples/test-C-1.png}
    \caption{EMNIST example two, a test sample labeled C}
    \label{fig:ex2}
\end{figure}

\begin{table}[H]
    \renewcommand{\arraystretch}{1.3}
    \caption{Explainability matrix for EMNIST example two.}\label{table:emnist_example2_ex}
    %\centering
    \begin{center}
    \resizebox{\textwidth}{!}{%
    \begin{tabular}{| c | c | c | c | c | c | c || c | c | c | c |}
    \cline{4-11}
    \multicolumn{3}{c}{} & \multicolumn{4}{|c||}{Effectiveness} & \multicolumn{4}{c|}{Explainability} \\
    \hline
    $F_j$ & Property & Vote & $E(j,C)$ & $E(j,T)$ & $E(j,U)$ & $E(j,X)$ & $Ex(C)$ & $Ex(T)$ & $Ex(U)$ & $Ex(X)$ \\
    \hline
    \hline
    $F_0$ & Stroke & C & $0.964$ &  &  &  & $1.0$ & & & \\ 
    \hline
    $F_1$ & Circle & C & $0.114$ &  &  &  & $1.0$ & & & \\
    \hline
    $F_2$ & Crossing & C & $0.056$ &  &  &  & $1.0$ & & &  \\
    \hline
    $F_3$ & Ellipse & T &  & $0.009$ &  &  &  & $1.0$ & & \\
    \hline
    $F_4$ & Ell-Cir & C & $0.131$ &  &  &  & $1.0$ & & & \\
    \hline
    $F_5$ & Endpoint & C & $0.574$ &  &  &  & $1.0$ & & &  \\
    \hline
    $F_6$ & Flood Fill & X &  &  &  & $0.005$ &  & & & $1.0$  \\
    \hline
    $F_7$ & Line & U &  &  & $0.244$ &  &  &  & $1.0$ & \\
    \hline
    $F_8$ & Convex Hull & C & $0.603$ &  &  &  & $1.0$ & & & \\
    \hline
    $F_9$ & Corner & C & $0.369$ &  &  &  & $1.0$ & & &  \\
    \hline
    $F_{10}$ & Unexplainable & C & $0.989$ &  &  &  & $0.0$ & & &  \\
    \hline
    \hline
    \multicolumn{3}{|c|}{Weights - $WE(c)$ / $\sum E(j,c)X_j$} & $3.801$ & $0.009$ & $0.244$ & $0.005$ & $2.812$ & $0.009$ & $0.244$ & $0.005$ \\
    \cline{0-10}
    \multicolumn{3}{|c|}{Confidence/Explainability} & $93.6\%$ & $0.02\%$ & $6.02\%$ & $0.01\%$ & $74.0\%$ & $100\%$ & $100\%$ & $100\%$ \\
    \cline{0-10}
    \end{tabular}%
    }
    \end{center}
    \label{table:example2}
\end{table}

Table \ref{table:emnist_example2_explanation} shows the explanation for Example 2.

\begin{table}[H]
    \renewcommand{\arraystretch}{1.3}
    \captionof{table}{Explanation for EMNIST Example two} \label{table:emnist_example2_explanation}
    %\centering
    \begin{center}
    %\resizebox{\columnwidth}{!}{%
    \begin{tabular}{| m{0.06\linewidth} | m{0.14\linewidth} | m{0.17\linewidth} | m{0.55\linewidth} |}
    \hline
     Class & Confidence & Explainability & Explainable Description \\
    \hline \hline
    C & $93.6\%$ & $74.0\%$ & Confidence is high for interpreting this character as a C due to the stroke, convex hull, endpoint, corner, ellipse-circle, circle, and crossing properties. \\ 
    \hline
    U & $6.02\%$ & $100\%$ & Confidence is low for interpreting this character as a U due to the line property. \\
    \hline
    T & $0.02\%$ & $100\%$ & Confidence is low for interpreting this character as a T due to the ellipse property. \\
    \hline
    X & $0.01\%$ & $100\%$ & Confidence is low for interpreting this character as an X due to the enclosed region property. \\ 
    \hline
    \end{tabular}
    %}
    \end{center}
\end{table}

The vote for an uppercase X, by the enclosed region, is unexpected as there
should be no resulting enclosed pixels in the transformed C, i.e., a flood fill,
which is the transform associated with the enclosed region, should result in no
pixels in the output image since a C has no enclosed region. A review of the
resulting transform shows that this is indeed the case.  Test results show that
samples from several classes, such as Y and L, that would not have pixels in a
flood fill, get Z votes from the enclosed region property.  This suggests that a
future improvement in the system could be to discount results from transforms
void of any activated pixels, especially when the transform has several classes
that generate empty images.  Improvements could also be accomplished using
knowledge of the explainable properties associated and not associated with each
class.


\subsection{Effectiveness Metrics}

This section presents another explainable example form EMNIST in the context of
differing effectiveness metrics.  The example illustrates the confidence,
rationale, and explainability with three effectiveness metrics. Data from the
system using accuracy, recall, and $E_{PARS}$, will be presented to demonstrate
how the explainable system benefits from more robust and resilient effectiveness
metrics.

Recall that effectiveness characterizes the performance of an inference engine
to predict a particular class and is used in the decision-making process to
select among the classes suggested. Even though the training and testing
datasets used are balanced, with respect to elements in each class, taking the
effectiveness by class results in comparing one class to all others. For
example, in the case of MNIST with ten balanced digit classes, the results of
one versus others for effectiveness has the others group with nine times the
number of elements of the single class.

\begin{figure}[H]
    \centering
    \includegraphics[width=2cm, alt={An EMNIST character labeled as an uppercase S}]{./images/emnist/test/18_28_0.png}
    \caption{EMNIST example three, a test sample labeled the uppercase letter S.}
    \label{fig:ex4}
\end{figure}

Figure~\ref{fig:ex4} depicts EMNIST example three from the balanced test set,
(index 18) labeled as an uppercase S. The explainable architecture used for the
example is composed of eleven PDFs with SVM IEs. The entire EMNIST balanced
dataset with digits, uppercase and lowercase characters was used for training
rather than the splits previously observed.

The results of processing the example using the accuracy metric are
depicted in Tables \ref{table:example4a} and \ref{table:exexample4a}, the recall
metric results are in Tables \ref{table:example4r} and \ref{table:exexample4r},
and the results of using the $E_{PARS}$ metric are in Tables
\ref{table:example4epars} and \ref{table:exexample4epars}.  Table
\ref{table:example4eff_metrics} illustrates the metric particulars for the PDFs
in the example in one place.

It is important to note that four classes () were voted on by the flows for EMNIST
example three.  Each PDF's vote and explainability metric is the same across the
various tables. Only the effectiveness varies as we change the metric from
accuracy, to recall, and finally to $E_{PARS}$.

%accuracy
\begin{table}[H]
    \renewcommand{\arraystretch}{1.3}
    \captionof{table}{Explainability Matrix for EMNIST example three based on the Accuracy metric.}\label{table:example4a}
    %\centering
    \begin{center}
    \resizebox{\textwidth}{!}{%
    \begin{tabular}{| c | c | c | c | c | c | c || c | c | c | c |}
    \cline{4-11}
    \multicolumn{3}{c}{} & \multicolumn{4}{|c||}{PDF Effectiveness} & \multicolumn{4}{c|}{PDF Explainability} \\
    \hline
    $F_j$ & Property & Class Vote & $E(j,1)$ & $E(j,E)$ & $E(j,S)$ & $E(j,n)$ & $Ex(1)$ & $Ex(E)$ & $Ex(S)$ & $Ex(n)$ \\
    \hline
    \hline
    $F_1$ & Stroke & S &  &  & $0.996$ &  &  &  & $1.0$ & \\ 
    \hline
    $F_2$ & Circle & 1 & $0.842$ &  &  &  & $1.0$ &  &  & \\
    \hline
    $F_3$ & Crossing & 1 &  $0.735$ &  &  &  & $1.0$ &  &  & \\
    \hline
    $F_4$ & Ellipse & n &  &  &  & $0.977$ &  &  &  & $1.0$ \\
    \hline
    $F_5$ & Ell-Cir & 1 & $0.844$ &  &  &  & $1.0$ &  &  & \\
    \hline
    $F_6$ & Endpoint & S &  &  & $0.978$ &  &  &  & $1.0$ & \\
    \hline
    $F_7$ & Flood Fill & 1 & $0.362$ &  &  &  & $1.0$ &  &  & \\
    \hline
    $F_8$ & Line & S &  &  & $0.979$ &  &  &  & $1.0$ & \\
    \hline
    $F_9$& Convex Hull & E &  & $0.977$ &  &  &  & $1.0$ &  &  \\
    \hline
    $F_{10}$& Corner & S &  &  & $0.979$ &  &  &  & $1.0$ & \\
    \hline
    $F_{11}$& Unexplainable & S &  &  & $0.996$ &  &  &  & $0.0$ & \\
    \hline
    \hline
    \multicolumn{3}{|c|}{Weights} & $2.783$ & $0.977$ & $4.928$ & $0.977$ & $2.783$ & $0.977$  & $3.932$ & $0.977$ \\
    \cline{0-10}
    \multicolumn{3}{|c|}{Confidence / Explainability} & $28.8\%$ & $10.1\%$ & $51.0\%$ & $10.1\%$ & $100\%$ & $100\%$ & $79.8\%$ & $100\%$ \\
    \cline{0-10}
    \end{tabular}
    }
    \end{center}
\end{table}

\begin{table}[H]
    \renewcommand{\arraystretch}{1.3}
    \captionof{table}{Explanation for EMNIST example three based on the Accuracy metric.} \label{table:exexample4a}
    %\centering
    \begin{center}
    %\resizebox{\columnwidth}{!}{%
    \begin{tabular}{| m{0.06\linewidth} | m{0.14\linewidth} | m{0.17\linewidth} | m{0.55\linewidth} |}
        \hline
        Class & Confidence & Explainability & Explainable Description \\
        \hline \hline
        S & $51.0\%$ & $79.8\%$ & Confidence is medium for interpreting this character as an S due to the stroke, corner, endpoint, and line properties. \\ 
        \hline
        1 & $28.8\%$ & $100\%$ & Confidence is medium for interpreting this character as a one due to the enclosed region, circle, ellipse-circle, and crossing properties. \\
        \hline
        n & $10.1\%$ & $100\%$ & Confidence is low for interpreting this character as an n due to the ellipse property. \\
        \hline
        E & $10.1\%$ & $100\%$ & Confidence is low for interpreting this character as an E due to the convex hull property. \\
       \hline
    \end{tabular}
    %}
    \end{center}
\end{table}

When examining the Explainability data from Tables \ref{table:example4r},
\ref{table:example4a}, and \ref{table:example4epars}, each flow identifier,
$F_j$, is in the first column where $j$ denotes the flow number.  The second
column indicates the property name. Flow $F_{11}$ is labeled as Unexplainable
because it represents an unexplainable flow, without an explainable property.
The Class Vote column indicates the class that was selected by each flow.

The remaining columns are related to effectiveness and explainability.  Since
four classes were voted on in this example, there will be four columns each for
effectiveness and explainability, representing each class with a vote. The
columns labeled $E(j,c)$, represent the effectiveness of the flows for each
class, $c$, that were voted upon. The columns labeled $Ex(c)$ represent the
explainability metric for each flow, $j$, and class, $c$, the flow voted for.
Only the column representing the class that a flow voted for will have a value
in the table for effectiveness and explainability.  The last two rows of the
Explainability tables represent the weights, of effectiveness and
explainability, as well as the Confidence, $Conf(c)$, and Explainability,
$Ex(c)$, for each class, $c$, and flows suggesting that class.

Ordered explanation results are represented in Tables \ref{table:exexample4r},
\ref{table:exexample4a}, and \ref{table:exexample4epars}.  The first column
indicates the class, $d$. The second column the Confidence.  The third column
represents the explainability associated with the decision. The final column is
the rationale provided to the user.

\begin{table}[H]
        \renewcommand{\arraystretch}{1.3}
        \captionof{table}{Explainability matrix for EMNIST example three based on the Recall metric.}\label{table:example4r}
        %\centering
        \begin{center}
        \resizebox{\textwidth}{!}{%
        \begin{tabular}{| c | c | c | c | c | c | c || c | c | c | c |}
        \cline{4-11}
        \multicolumn{3}{c}{} & \multicolumn{4}{|c||}{PDF Effectiveness} & \multicolumn{4}{c|}{PDF Explainability} \\
        \hline
        $F_j$ & Property & Class Vote & $E(j,1)$ & $E(j,E)$ & $E(j,S)$ & $E(j,n)$ & $Ex(1)$ & $Ex(E)$ & $Ex(S)$ & $Ex(n)$ \\
        \hline
        \hline
        $F_1$ & Stroke & S &  &  & $0.923$ &  &  &  & $1.0$ & \\ 
        \hline
        $F_2$ & Circle & 1 & $0.995$ &  &  &  & $1.0$ &  &  & \\
        \hline
        $F_3$ & Crossing & 1 & $0.985$ &  &  &  & $1.0$ &  &  & \\
        \hline
        $F_4$ & Ellipse & n &  &  &  & $0.108$ &  &  &  & $1.0$ \\
        \hline
        $F_5$ & Ell-Cir & 1 & $0.995$ &  &  &  & $1.0$ &  &  & \\
        \hline
        $F_6$ & Endpoint & S &  &  & $0.496$ &  &  &  & $1.0$ & \\
        \hline
        $F_7$ & Flood Fill & 1 & $0.999$ &  &  &  & $1.0$ &  &  & \\
        \hline
        $F_8$ & Line & S &  &  & $0.495$ &  &  &  & $1.0$ & \\
        \hline
        $F_9$& Convex Hull & E &  & $0.260$ &  &  &  & $1.0$ &  &  \\
        \hline
        $F_{10}$& Corner & S &  &  & $0.509$ &  &  &  & $1.0$ & \\
        \hline
        $F_{11}$& Unexplainable & S &  &  & $0.924$ &  &  &  & $0.0$ & \\
        \hline
        \hline
        \multicolumn{3}{|c|}{Weights} & $3.974$ & $0.260$ & $3.347$ & $0.108$ & $3.974$ & $0.260$  & $2.423$ & $0.1078$ \\
        \cline{0-10}
        \multicolumn{3}{|c|}{Confidence / Explainability} & $51.7\%$ & $3.39\%$ & $43.5\%$ & $1.40\%$ & $100\%$ & $100\%$ & $72.4\%$ & $100\%$ \\
        \cline{0-10}
        \end{tabular}
        }
        \end{center}
\end{table}
    
\begin{table}[H]
    \renewcommand{\arraystretch}{1.3}
    \captionof{table}{Explanation for EMNIST example three based on the Recall metric.} \label{table:exexample4r}
    %\centering
    \begin{center}
    %\resizebox{\columnwidth}{!}{%
    \begin{tabular}{| m{0.06\linewidth} | m{0.14\linewidth} | m{0.17\linewidth} | m{0.55\linewidth} |}
        \hline
         Class & Confidence & Explainability & Explainable Description \\
        \hline \hline
        1 & $51.7\%$ & $100\%$ & Confidence is medium for interpreting this character as a one due to the enclosed region, circle, ellipse-circle, and crossing properties. \\ 
        \hline
        S & $43.5\%$ & $72.4\%$ & Confidence is medium for interpreting this character as an S due to the stroke, corner, endpoint, and line properties. \\
        \hline
        E & $3.39\%$ & $100\%$ & Confidence is low for interpreting this character as an E due to the convex hull property. \\
        \hline
        n & $1.40\%$ & $100\%$ & Confidence is low for interpreting this character as an n due to the ellipse property. \\
        \hline
    \end{tabular}
    %}
    \end{center}
\end{table}

Observe in Tables \ref{table:example4r} and \ref{table:exexample4r} that the
four PDFs that voted for class one had high Recall, reflected in the
Effectiveness column $E(j,1)$ of Table \ref{table:example4r}.  This resulted in
the digit one winning with medium confidence, $51.7\%$, due to the enclosed
region, circle, ellipse-circle, and crossing properties.  The second choice was
the \lq{S}\rq~with medium confidence, $43.5\%$. Since Recall in
\eqref{eq:recall} has $TP$ in the numerator, high $TP$ and very low $FN$ counts
of the PDFs that voted for the digit one observed in Table
\ref{table:example4eff_metrics} explains why the digit one wins using the Recall
metric. Observe that the Recall metrics for each flow that voted for one is near
or above $99\%$.

Tables \ref{table:example4a} and \ref{table:exexample4a} contain data and
results from using the Accuracy metric on the example.  Accuracy is given in
\eqref{eq:accuracy}.  Due to the comparatively high $TN$ rate of the PDFs that
voted for the \lq{S}\rq, as shown in bold in Table
\ref{table:example4eff_metrics} it wins with medium confidence, $51.0\%$, due to
the stroke, corner, endpoint, and line properties. The \lq{S}\rq~also received a
vote from the unexplainable PDF.  The lower explainability metric, at $79.8\%$,
reflects diminished explainability due to no explainable property associated
with part ($20.2\%$) of the contribution for the decision.

% product
\begin{table}[H]
    \renewcommand{\arraystretch}{1.3}
    \captionof{table}{Explainability Matrix for EMNIST example three based on the $E_{PARS}$ metric.}\label{table:example4epars}
    %\centering
    \begin{center}
    \resizebox{\textwidth}{!}{%
    \begin{tabular}{| c | c | c | c | c | c | c || c | c | c | c |}
    \cline{4-11}
    \multicolumn{3}{c}{} & \multicolumn{4}{|c||}{PDF Effectiveness} & \multicolumn{4}{c|}{PDF Explainability} \\
    \hline
    $F_j$ & Property & Class Vote & $E(j,1)$ & $E(j,E)$ & $E(j,S)$ & $E(j,n)$ & $Ex(1)$ & $Ex(E)$ & $Ex(S)$ & $Ex(n)$ \\
    \hline
    \hline
    $F_1$ & Stroke & S &  &  & $0.834$ &  &  &  & $1.0$ & \\ 
    \hline
    $F_2$ & Circle & 1 & $0.083$ &  &  &  & $1.0$ &  &  & \\
    \hline
    $F_3$ & Crossing & 1 & $0.039$ &  &  &  & $1.0$ &  &  & \\
    \hline
    $F_4$ & Ellipse & n &  &  &  & $0.041$ &  &  &  & $1.0$ \\
    \hline
    $F_5$ & Ell-Cir & 1 & $0.085$ &  &  &  & $1.0$ &  &  & \\
    \hline
    $F_6$ & Endpoint & S &  &  & $0.227$ &  &  &  & $1.0$ & \\
    \hline
    $F_7$ & Flood Fill & 1 & $0.004$ &  &  &  & $1.0$ &  &  & \\
    \hline
    $F_8$ & Line & S &  &  & $0.240$ &  &  &  & $1.0$ & \\
    \hline
    $F_9$& Convex Hull & E &  & $0.107$ &  &  &  & $1.0$ &  &  \\
    \hline
    $F_{10}$& Corner & S &  &  & $0.251$ &  &  &  & $1.0$ & \\
    \hline
    $F_{11}$& Unexplainable & S &  &  & $0.839$ &  &  &  & $0.0$ & \\
    \hline
    \hline
    \multicolumn{3}{|c|}{Weights} & $0.211$ & $0.107$ & $2.391$ & $0.041$ & $0.2106$ & $0.107$  & $1.552$ & $0.041$ \\
    \cline{0-10}
    \multicolumn{3}{|c|}{Confidence  / Explainability} & $7.66\%$ & $3.88\%$ & $86.9\%$ & $1.48\%$ & $100\%$ & $100\%$ & $64.9\%$ & $100\%$ \\
    \cline{0-10}
    \end{tabular}
    }
    \end{center}
\end{table}

\begin{table}[H]
    \renewcommand{\arraystretch}{1.3}
    \captionof{table}{Explanations for EMNIST example three based on the $E_{PARS}$ metric.} \label{table:exexample4epars}
    %\centering
    \begin{center}
    %\resizebox{\columnwidth}{!}{%
    \begin{tabular}{| m{0.06\linewidth} | m{0.14\linewidth} | m{0.17\linewidth} | m{0.55\linewidth} |}
    \hline
     Class & Confidence & Explainability & Explainable Description \\
    \hline \hline
    S\rq & $86.9\%$ & $64.9\%$ & Confidence is high for interpreting this character as an S due to the stroke, corner, endpoint, and line properties. \\ 
    \hline
    1 & $7.66\%$ & $100\%$ & Confidence is low for interpreting this character as a one due to the enclosed region, circle, ellipse-circle, and crossing properties. \\
    \hline
    E & $3.88\%$ & $100\%$ & Confidence is low for interpreting this character as an E due to the convex hull property. \\
    \hline
    n & $1.40\%$ & $100\%$ & Confidence is low for interpreting this character as an n due to the ellipse property. \\
    \hline
    \end{tabular}
    %}
    \end{center}
\end{table}   

Table~\ref{tab:mnist_emnist_eff_metrics} depicts the MNIST and EMNIST overall
accuracy percentage results obtained with MLP and SVM using various per-class
effectiveness metrics in ranking and selecting a global decision. MLPs and SVMs
performed comparably on MNIST and the SVMs performed a few percentage points
better on EMNIST.  SVMs also appeared to be more forgiving when used with lower
performing effectiveness metrics. 

Accuracy reflected in the tables is an appropriate overall measure of
performance of the system since the classes are balanced and the metric is being
taken on the entire architecture. The Explainable result (E) columns indicate
the overall system accuracy for ten explainable flows while the combined
Explainable and Unexplainable result (E+U) columns indicate the overall
accuracy using ten explainable and one unexplainable flow.

\begin{table}[H]
    \renewcommand{\arraystretch}{1.3}
    %\centering
    \caption{Comparison of an explainable system's overall performance (\% accuracy) using various effectiveness metrics.}
    \begin{center}
    %\resizebox{\columnwidth}{!}{%
    \begin{tabular}{| l | c | c || c | c || c | c || c | c |}
        \cline{2-9}
        \multicolumn{1}{l}{} & \multicolumn{4}{| c ||}{MNIST} & \multicolumn{4}{ c |}{EMNIST} \\
        \cline{2-9}
        \multicolumn{1}{l}{} & \multicolumn{2}{| c ||}{MLP} & \multicolumn{2}{ c ||}{SVM} & \multicolumn{2}{ c ||}{MLP} & \multicolumn{2}{ c |}{SVM} \\
        \hline
        Effectiveness Metric & $E$ & $E+U$ & $E$ & $E+U$ & $E$ & $E+U$ & $E$ & $E+U$ \\
        \hline
        \hline
        $E_{PARS}$ & 95.5 & 97.6 & 95.4 & 97.3 & 71.7 & 77.4 & 75.9 & 81.0 \\ \hline
        $P \cdot R \cdot S$ & 95.2 & 97.4 & 95.1 & 97.1 & 71.7 & 77.3 & 75.9 & 81.0 \\ \hline
        $P \cdot R$ & 94.3 & 97.2 & 94.8 & 97.0 & 71.1 & 76.8 & 75.9 & 81.0 \\ \hline
        $P \cdot S$ & 95.0 & 97.3 & 94.4 & 96.8 & 70.4 & 76.9 & 74.7 & 80.8 \\ \hline
        Precision ($P$) & 94.4 & 97.0 & 94.2 & 96.7 & 70.4 & 76.8 & 74.7 & 80.8 \\ \hline
        Cohen's Kappa & 94.1 & 96.9 & 94.2 & 96.7 & 71.5 & 77.5 & 74.4 & 80.6 \\ \hline
        $MCC$ & 92.6 & 96.2 & 93.8 & 96.5 & 69.7 & 75.5 & 73.9 & 80.4 \\ \hline
        F1-Score & 91.9 & 95.9 & 93.5 & 96.4 & 70.6 & 76.7 & 74.2 & 80.5 \\ \hline
        $S \cdot R$ & 85.6 & 93.1 & 92.0 & 95.7 & 37.0 & 55.3 & 67.1 & 76.9 \\ \hline
        Specificity ($S$) & 88.1 & 93.7 & 92.0 & 95.1 & 49.2 & 61.5 & 68.5 & 76.9 \\ \hline
        Accuracy (ACC) & 85.6 & 92.6 & 92.0 & 95.1 & 47.7 & 60.0 & 68.0 & 76.3 \\ \hline
        AUC & 67.5 & 77.7 & 89.8 & 94.1 & 5.62 & 8.49 & 63.4 & 73.1 \\ \hline
        Balanced Accuracy & 75.0 & 85.4 & 89.7 & 94.2 & 8.46 & 16.4 & 63.1 & 72.9 \\ \hline
        Recall ($R$) & 52.4 & 68.0 & 85.1 & 91.7 & 2.53 & 3.35 & 50.5 & 63.7 \\
        \hline
    \end{tabular}
    %}
    \end{center}
    \label{tab:mnist_emnist_eff_metrics}
\end{table}

Adding an accurate, but unexplainable, classifier to the system improves
performance. This is illustrated in Table \ref{tab:mnist_emnist_eff_metrics}
where combined results (E+U) are greater than the strictly explainable ($E$)
results. A marked improvement in accuracy was shown by moving to combined
explainable and unexplainable flows.

The combined (E+U) results were greater than explainable (E) in all cases.
In the MNIST example using Recall as Effectiveness, the increase using MLP flows
was over 15\%.  Typical increases with combined flows ranged from two to ten
percentage points better than strictly explainable flows.

Some of the performance metrics from the literature, especially those that are
resilient to imbalanced data such as Cohen's Kappa, provide outstanding results
as a measure of effectiveness in the explainable architecture. Surprisingly,
Precision alone as well as the product of Precision and other metrics perform
among the highest of those attempted. The best-performing metric found is given
in \eqref{eq:epars} as $E_{PARS}$, the product of Precision, Accuracy, Recall,
and Specificity.  The previously reported accuracy observed, using the
explainable architecture, on MNIST with MLPs was about 92\% using Recall as
Effectiveness and probability estimates. Employing $E_{PARS}$ as effectiveness
and without using probability estimates, accuracy was observed at about 95.5\%,
an improvement of over 3\%. When probability estimates are used with $E_{PARS}$
as effectiveness, accuracy was increased to 96.1\%.  Adding an unexplainable
component to the system and using probability estimates increased accuracy
results to as high as 98.0\%. 

Observing the numerator in \eqref{eq:epars_expansion}, $E_{PARS}$
performs well because $TP$ to $TN$ balance is maintained since $TP^3$
($\approx2.2\text{x}10^{11}$ for MNIST and $\approx1.4\text{x}10^{10}$ for
EMNIST) and $TN^2$ ($\approx2.7\text{x}10^9$ for MNIST and
$\approx1.2\text{x}10^{10}$ for EMNIST).  A larger dataset with more classes may
not result in similar performance.

\begin{table}[H]
    \renewcommand{\arraystretch}{1.3}
    \captionof{table}{Ex.~2 Effectiveness Metrics (\%) based on Votes}\label{table:example4eff_metrics} 
    %\centering
    \begin{center}
    %\resizebox{\columnwidth}{!}{%
    \begin{tabular}{| c | c | c | c | c | c | c | c | c | c | c | c | c | c |}
    \hline
    Property & Class & TP & TN & FP & FN & R & ACC & $E_{PARS}$ \\
    \hline
    \hline
    Flood Fill & 1 & 2.13 & 34.0 & 63.8 & 0.0 & 99.9 & 36.2 & 0.40 \\ 
    \hline
    Circle & 1 & 2.12 & 82.1 & 15.8 & 0.0 & 99.5 & 84.2 &  8.29 \\
    \hline
    Ell-Cir & 1 & 2.12 & 82.3 & 15.5 & 0.0 & 99.5 & 84.4 & 8.47 \\
    \hline
    Crossing & 1 & 2.01 & 71.5 & 26.4 & 0.0 & 98.5 & 73.5 & 3.88 \\
    \hline
    Convex Hull & E & 0.55 & 97.1 & 0.76 & 1.57 & 26.0 & 97.7 & 10.7 \\
    \hline
    Unexplainable & S & 1.97 & \bf{97.7} & 0.19 & 0.16 & 92.4 & 99.6 & 83.9 \\
    \hline
    Stroke & S & 1.96 & \bf{97.7} & 0.20 & 0.16 & 92.3 & 99.6 & 83.4 \\
    \hline
    Corner & S & 1.08 & \bf{96.8} & 1.04 & 1.05 & 50.9 & 97.9 & 25.1 \\
    \hline
    Endpoint & S & 1.06 & \bf{96.7} & 0.20 & 0.20 & 49.6 & 97.8 & 22.7 \\
    \hline
    Line & S & 1.05 & 96.8 & 1.05 & 1.07 & 49.5 & 97.9 & 24.0 \\
    \hline
    Ellipse & n & 0.23 & 97.5 & 0.36 & 1.90 & 10.8 & 97.7 & 4.07 \\
    \hline
    \end{tabular}
    %}
    \end{center}
\end{table}

The final set of Explainability and Explanations Tables,
\ref{table:example4epars} and \ref{table:exexample4epars}, involve the
$E_{PARS}$ metric. The stroke and unexplainable flows have a much higher
effectiveness than other metrics.  High effectiveness from the Accuracy
metric for the flows voting for the digit one were due to insensitivity of
Accuracy to $FP$s.  The same flows voting for the digit one have a reduced
effectiveness due to relative sensitivity of $E_{PARS}$ to $FP$ counts, as
observed in Table \ref{table:example4eff_metrics}.  The capital \lq{S}\rq~
correctly wins with high confidence, $86.9\%$, using $E_{PARS}$.


\subsection{Pruning Mislabeled EMNIST Data}

Table~\ref{tab:raw_cap_confusion_matrix} shows the confusion matrix for interesting
uppercase letters after running though the explainable architecture. There is
significant confusion with the I and L classes and a fair degree of confusion
with the U and V classes.

%\begin{figure}[h]
%    \centering
%    \includegraphics[width=10cm]{./images/confusion_matrix_vlsid.png}
%    \caption{Confusion matrix using convex hull}
%    \label{fig:conf_matrix}
%\end{figure}

% \begin{table}[H]
%     \renewcommand{\arraystretch}{1.3}
%     \centering
%     \caption{Explainable confusion matrix for unpruned uppercase letters H - M, U, and V.  TODO: this needs to be formatted like other confusion matrices}
%     %\resizebox{\columnwidth}{!}{%
%     \begin{tabular}{ |c|c|c|c|c|c|c|c|c|}
%     \hline
%      & H & I & J & K & L & M & U & V \\
%     \hline
%     H & 374 &  & 1 & 3 &  & 11 & 1 & 1 \\
%     \hline
%     I &  & 275 & 2 & & 114 & & & \\
%     \hline
%     J &  & 11 & 376 & & 1 & & 2 & 6 \\
%     \hline
%     K &  &  &  & 392 & 1 & & & \\
%     \hline
%     L &  & 90 &  &  & 306 & & & \\
%     \hline
%     M &  &  &  &  &  & 399 & & 1 \\
%     \hline
%     U & 4 & 1 & & 1 & 9 & & 305 & 29 \\
%     \hline
%     V & & & 2 & & 2 & 2 & 9 & 326 \\
%     \hline
%     \end{tabular}
%     %}
%     \label{raw_cap_confusion_matrix}
% \end{table}


\begin{table}[H]
    %\centering
    \caption{Explainable confusion matrix for select unpruned uppercase letters.}
    \begin{center}
    \label{tab:raw_cap_confusion_matrix}
    \renewcommand{\arraystretch}{1.3}
    \begin{tabular}{ll|c|c|c|c|c|c|c|c|}
        \multicolumn{2}{c}{}& \multicolumn{8}{c}{Predicted}\\
        & \multicolumn{1}{c}{} & \multicolumn{1}{c}{H} & \multicolumn{1}{c}{I} & \multicolumn{1}{c}{J}
        & \multicolumn{1}{c}{K} & \multicolumn{1}{c}{L} & \multicolumn{1}{c}{M} & \multicolumn{1}{c}{U}
        & \multicolumn{1}{c}{V} \\
        \cline{3-10}
        \multirow{8}{*}{{\rotatebox[origin=c]{90}{Actual}
        }} & 
        H &     374 &  0   &  1   &  3  &  0  &  11  &  1  &  1  \\ \cline{3-10}
        &   I &  0  &  275 &  2   &  0  &  114  &  0  &  0  & 0  \\ \cline{3-10}
        &   J &  0  &  11  &  376 &  0  &  1  &  0  &  2  &  6   \\ \cline{3-10}
        &   K &  0  &  0   &  0   & 392 &  1  &  0  &  0  &  0   \\ \cline{3-10}
        &   L &  0  &  90  &  0   &  0  & 306 &  0  &  0  &  0   \\ \cline{3-10}
        &   M &  0  &  0   &  0   &  0  &  0  & 399 &  0  &  1   \\ \cline{3-10}
        &   U &  4  &  1   &  0   &  1  &  9  &  0  & 305 &  29  \\ \cline{3-10}
        &   V &  0  &  0   &  2   &  0  &  2  &  2  &  9  &  326 \\ \cline{3-10}
    \end{tabular}
    \end{center}
\end{table}


After applying one round of Cleanlab to prune only training labels with issues,
Table \ref{tab:raw_cap_cleanlab_confusion_matrix} indicates the same uppercase
letters of interest. The confusion in the I class appears to have gotten better.
L, U, and V classes appear to have gotten worse. Six of the eight classes of
interest had decreases in correct classifications, the values along the
diagonal.  Using CNNs in the explainable architecture, the overall results of
one round of Cleanlab were an increase in accuracy to 87.4\% on uppercase
letters from 82.9\% without pruning.

% \begin{table}[H]
%     \renewcommand{\arraystretch}{1.3}
%     \centering
%     \caption{Explainable confusion matrix for uppercase letters H - M, U, and V after a single Cleanlab pruning.  TODO this confusion matrix needs to be formatted like others}
%     %\resizebox{\columnwidth}{!}{%
%     \begin{tabular}{ |c|c|c|c|c|c|c|c|c|}
%     \hline
%     ~ & H & I & J & K & L & M & U & V \\
%     \hline
%     H & 368 & & 1 & 2 & & 7 & 1 & \\
%     \hline
%     I &  & 293 & 4 &  & 65 & 14 & & \\
%     \hline
%     J & & 17 & 339 & & 1 & 6 & 1 & 2 \\
%     \hline
%     K & 5 &  &  & 349 & 2 & 20 &  & \\
%     \hline
%     L & 0 & 120 &  & 1 & 236 & 2 & & \\
%     \hline
%     M & 4 & & & 3 & & 383 & & \\
%     \hline
%     U & 2 & & 1 & 1 & & 22 & 304 & 17 \\
%     \hline
%     V & & & 3 & 1 & & 21 & 13 & 315 \\
%     \hline
%     \end{tabular}
%     %}
%     \label{raw_cap_cleanlab_confusion_matrix}
% \end{table}

\begin{table}[H]
    %\centering
    \caption{Explainable confusion matrix for select uppercase letters after a single round of Cleanlab pruning.}
    \begin{center}
    \label{tab:raw_cap_cleanlab_confusion_matrix}
    \renewcommand{\arraystretch}{1.3}
    \begin{tabular}{ll|c|c|c|c|c|c|c|c|}
        \multicolumn{2}{c}{}& \multicolumn{8}{c}{Predicted}\\
        & \multicolumn{1}{c}{} & \multicolumn{1}{c}{H} & \multicolumn{1}{c}{I} & \multicolumn{1}{c}{J}
        & \multicolumn{1}{c}{K} & \multicolumn{1}{c}{L} & \multicolumn{1}{c}{M} & \multicolumn{1}{c}{U}
        & \multicolumn{1}{c}{V} \\
        \cline{3-10}
        \multirow{8}{*}{{\rotatebox[origin=c]{90}{Actual}
        }} & 
        H &     368 &  0   &  1   &  2  &  0  &  7  &  1  &  0  \\ \cline{3-10}
        &   I &  0  &  293 &  4   &  0  & 65 &  14  &  0  &  0  \\ \cline{3-10}
        &   J &  0  &  17  & 339  &  0  & 1  &  6  &  1  &  2   \\ \cline{3-10}
        &   K &  5  &  0   &  0   & 349 &  2  &  20  &  0  &  0   \\ \cline{3-10}
        &   L &  0  &  120 &  0   &  1  & 236 &  2  &  0  &  0   \\ \cline{3-10}
        &   M &  4  &  0   &  0   &  3  &  0  & 383 &  0  &  0   \\ \cline{3-10}
        &   U &  2  &  0   &  1   &  1  &  0  &  22  & 304 &  17  \\ \cline{3-10}
        &   V &  0  &  0   &  3   &  1  &  0  &  21  &  13  &  315 \\ \cline{3-10}
    \end{tabular}
    \end{center}
\end{table}


After applying subsequent rounds of Cleanlab to the capital letter dataset, the
accuracy achieved was 86.6\% after the second and 88.7\% after the third
iteration of pruning. Table \ref{tab:raw_cap_cleanlab_third_confusion_matrix} again
shows the confusion matrix for the classes of interest after the third Cleanlab
pruning.

The best pruning results were achieved performing the 5\% pruning with a penalty
for incorrect labels.  The results on this pruned set were about 89.2\% accuracy
on uppercase letters versus 82.9\% on unpruned test data using CNN in the
explainable architecture.  

% \begin{table}[H]
%     \renewcommand{\arraystretch}{1.3}
%     \centering
%     \caption{Explainable confusion matrix for uppercase letters H - M, U, and V after three rounds of Cleanlab pruning}
%     %\resizebox{\columnwidth}{!}{%
%     \begin{tabular}{ |c|c|c|c|c|c|c|c|c|}
%     \hline
%     ~ & H & I & J & K & L & M & U & V \\
%     \hline
%     H & 357 & &  & 8 & & 2 & 1 & 1 \\
%     \hline
%     I &  & 254 & 3 & 3 & 101 & 1 & & \\
%     \hline
%     J & & 19 & 298 & 1 & 16 & & 2 & 6 \\
%     \hline
%     K & 3 &  &  & 352 & 2 & 2 & 0 & \\
%     \hline
%     L & 0 & 90 & 1 & 1 & 277 & & & \\
%     \hline
%     M & & & & 3 & & 378 & & 1 \\
%     \hline
%     U & 1 & & 1 & 9 & & 1 & 305 & 29 \\
%     \hline
%     V & & 2 & & 2 & 2 & 1 & 9 & 326 \\
%     \hline
%     \end{tabular}
%     %}
%     \label{raw_cap_cleanlab_third_confusion_matrix}
% \end{table}

\begin{table}[H]
    %\centering
    \caption{Explainable confusion matrix for select uppercase letters after three rounds of Cleanlab pruning.}
    \begin{center}
    \label{tab:raw_cap_cleanlab_third_confusion_matrix}
    \renewcommand{\arraystretch}{1.3}
    \begin{tabular}{ll|c|c|c|c|c|c|c|c|}
        \multicolumn{2}{c}{}& \multicolumn{8}{c}{Predicted}\\
        & \multicolumn{1}{c}{} & \multicolumn{1}{c}{H} & \multicolumn{1}{c}{I} & \multicolumn{1}{c}{J}
        & \multicolumn{1}{c}{K} & \multicolumn{1}{c}{L} & \multicolumn{1}{c}{M} & \multicolumn{1}{c}{U}
        & \multicolumn{1}{c}{V} \\
        \cline{3-10}
        \multirow{8}{*}{{\rotatebox[origin=c]{90}{Actual}
        }} & 
        H &     357 &  0   &  0   &  8  &  0  &  2  &  1  &  1  \\ \cline{3-10}
        &   I &  0  &  254 &  3   &  3  & 101 &  1  &  0  &  0  \\ \cline{3-10}
        &   J &  0  &  19  & 298  &  1  & 16  &  0  &  2  &  6   \\ \cline{3-10}
        &   K &  3  &  0   &  0   & 352 &  2  &  2  &  0  &  0   \\ \cline{3-10}
        &   L &  0  &  90  &  1   &  1  & 277 &  0  &  0  &  0   \\ \cline{3-10}
        &   M &  0  &  0   &  0   &  3  &  0  & 378 &  0  &  1   \\ \cline{3-10}
        &   U &  1  &  0   &  1   &  9  &  0  &  1  & 305 &  29  \\ \cline{3-10}
        &   V &  0  &  2   &  0   &  2  &  2  &  1  &  9  &  326 \\ \cline{3-10}
    \end{tabular}
    \end{center}
\end{table}

Images pruned from the training set due to issues such as mislabeled or
ambiguous samples after one Cleanlab pruning were submitted back to the system.  
Of the 1188 pruned images only 129, or about 11\%, of the problematic labels from
the original training set were recognized as correctly labeled.  The remaining
89\% were recognized by the explainable architecture as other classes with
varying degrees of confidence. Some of the pruned images are depicted in
Figure~\ref{fig:pruned_inf_samples}.  In many cases the explainable architecture is
able to correctly inference, according to human recognition, mislabeled samples.

\begin{figure}[H]
    \centering
    %\textbf{Pruned Samples}\par\medskip 
    \begin{subfigure}{.20\columnwidth}
        \centering
        \includegraphics[width=.90\textwidth, alt={An EMNIST character}]{./images/issues/excluded-G-0-inferred-S.png}
        \caption{G label}
        \label{fig:inf_issue_G0}
    \end{subfigure}%
    \begin{subfigure}{.20\columnwidth}
        \centering
        \includegraphics[width=.90\textwidth, alt={An EMNIST character}]{./images/issues/excluded-T-5-inferred-u.png}
        \caption{T label}
        \label{fig:inf_issue_T5}
    \end{subfigure}%
    \begin{subfigure}{.20\columnwidth}
        \centering
        \includegraphics[width=.90\textwidth, alt={An EMNIST character}]{./images/issues/excluded-O-7-inferred-M.png}
        \caption{O label}
        \label{fig:inf_issue_O7}
    \end{subfigure}%
    \begin{subfigure}{.20\columnwidth}
        \centering
        \includegraphics[width=.90\textwidth, alt={An EMNIST character}]{./images/issues/excluded-W-3-inferred-M.png}
        \caption{W label}
        \label{fig:inf_issue_W3}
    \end{subfigure}%
    %\begin{subfigure}{.20\columnwidth}
    %    \centering
    %    \includegraphics[width=.90\textwidth, alt={An EMNIST character}]{./images/issues/excluded-O-10-inferred-F.png}
    %    \caption{O label}
    %    \label{fig:inf_issue_O10}
    %\end{subfigure}
    \caption{Pruned samples that were inferred by the system.}
    \label{fig:pruned_inf_samples}
\end{figure}

\begin{figure}[h]
    \centering
    %\textbf{Pruned Samples}\par\medskip 
    \begin{subfigure}{.20\columnwidth}
        \centering
        \includegraphics[width=.90\textwidth, alt={An EMNIST character}]{./images/issues/excluded-G-0-stroke.png}
        \caption{Stroke}
        %\label{fig:inf_issue_G0_stroke}
    \end{subfigure}%
    \begin{subfigure}{.20\columnwidth}
        \centering
        \includegraphics[width=.90\textwidth, alt={An EMNIST character}]{./images/issues/excluded-G-0-circle.png}
        \caption{Circles}
        %\label{fig:inf_issue_T5}
    \end{subfigure}%
    \begin{subfigure}{.20\columnwidth}
        \centering
        \includegraphics[width=.90\textwidth, alt={An EMNIST character}]{./images/issues/excluded-G-0-el-circle.png}
        \caption{Ellipse}
        %\label{fig:inf_issue_O7}
    \end{subfigure}%
    \begin{subfigure}{.20\columnwidth}
        \centering
        \includegraphics[width=.90\textwidth, alt={An EMNIST character}]{./images/issues/excluded-G-0-area.png}
        \caption{Area}
        %\label{fig:inf_issue_W3}
    \end{subfigure}%
    \begin{subfigure}{.20\columnwidth}
        \centering
        \includegraphics[width=.90\textwidth, alt={An EMNIST character}]{./images/issues/excluded-G-0-line.png}
        \caption{Line}
        %\label{fig:inf_issue_O10}
    \end{subfigure}
    \caption{Relevant transforms of Figure~\ref{fig:inf_issue_G0}.}
    \label{fig:g0_trans}
\end{figure}

Figure~\ref{fig:inf_issue_G0}, originally labeled as a G, was inferred correctly by
the explainable architecture as an S with 42\% confidence.  The next closest
prediction was a Z with 14\% confidence.  The explanation for the prediction was
due to the stroke, circle, area, and line properties being consistent with the
letter S. Examples of the contributing transforms are depicted in
Figure~\ref{fig:g0_trans}.

\begin{figure}[H]
    \centering
    %\textbf{Pruned Samples}\par\medskip 
    \begin{subfigure}{.20\columnwidth}
        \centering
        \includegraphics[width=.90\textwidth, alt={An EMNIST character}]{./images/issues/excluded-T-5-stroke.png}
        \caption{Stroke}
        %\label{fig:inf_issue_G0_stroke}
    \end{subfigure}%
    \begin{subfigure}{.20\columnwidth}
        \centering
        \includegraphics[width=.90\textwidth, alt={An EMNIST character}]{./images/issues/excluded-T-5-area.png}
        \caption{Area}
        %\label{fig:inf_issue_T5}
    \end{subfigure}%
    \begin{subfigure}{.20\columnwidth}
        \centering
        \includegraphics[width=.90\textwidth, alt={An EMNIST character}]{./images/issues/excluded-T-5-endpoint.png}
        \caption{Endpoint}
        %\label{fig:inf_issue_O7}
    \end{subfigure}%
    \begin{subfigure}{.20\columnwidth}
        \centering
        \includegraphics[width=.90\textwidth, alt={An EMNIST character}]{./images/issues/excluded-T-5-circle.png}
        \caption{Circle}
        %\label{fig:inf_issue_W3}
    \end{subfigure}%
    %\begin{subfigure}{.20\columnwidth}
    %    \centering
    %    \includegraphics[width=.90\textwidth, alt={An EMNIST character}]{./images/issues/excluded-G-0-line.png}
    %    \caption{Line}
    %    %\label{fig:inf_issue_O10}
    %\end{subfigure}
    \caption{Relevant transforms of Figure~\ref{fig:inf_issue_T5}.}
    \label{fig:t5_trans}
\end{figure}

Figure~\ref{fig:inf_issue_T5}, originally labeled as a T, was inferred correctly as
a U with 36\% confidence. The next closest prediction was a Z with 25\%
confidence.  The explanation was due to the stroke, area,
endpoint, and circle properties. Figure~\ref{fig:t5_trans}
depicts relevant transforms.

\begin{figure}[H]
    \centering
    %\textbf{Pruned Samples}\par\medskip 
    \begin{subfigure}{.20\columnwidth}
        \centering
        \includegraphics[width=.90\textwidth, alt={An EMNIST character}]{./images/issues/excluded-O-7-stroke.png}
        \caption{Stroke}
        %\label{fig:inf_issue_G0_stroke}
    \end{subfigure}%
    \begin{subfigure}{.20\columnwidth}
        \centering
        \includegraphics[width=.90\textwidth, alt={An EMNIST character}]{./images/issues/excluded-O-7-endpoint.png}
        \caption{Endpoint}
        %\label{fig:inf_issue_T5}
    \end{subfigure}%
    \begin{subfigure}{.20\columnwidth}
        \centering
        \includegraphics[width=.90\textwidth, alt={An EMNIST character}]{./images/issues/excluded-O-7-area.png}
        \caption{Area}
        %\label{fig:inf_issue_O7}
    \end{subfigure}%
    \begin{subfigure}{.20\columnwidth}
        \centering
        \includegraphics[width=.90\textwidth, alt={An EMNIST character}]{./images/issues/excluded-O-7-circle.png}
        \caption{Circle}
        %\label{fig:inf_issue_W3}
    \end{subfigure}%
    %\begin{subfigure}{.20\columnwidth}
    %    \centering
    %    \includegraphics[width=.90\textwidth, alt={An EMNIST character}]{./images/issues/excluded-O-7-fill.png}
    %    \caption{Enclosed Region}
    %    %\label{fig:inf_issue_O10}
    %\end{subfigure}
    \caption{Relevant transforms of Figure~\ref{fig:inf_issue_O7}.}
    \label{fig:o7_trans}
\end{figure}

Figure~\ref{fig:inf_issue_O7}, originally labeled as an O, was inferred correctly
as an M with 56\% confidence. The next closest prediction was a Z with 27\%
confidence.  Rationale for the decision are the stroke, endpoint, area,
and circle being consistent with an M. Figure~\ref{fig:o7_trans} depicts the
transforms that contributed to the decision.

\begin{figure}[H]
    \centering
    %\textbf{Pruned Samples}\par\medskip 
    \begin{subfigure}{.20\columnwidth}
        \centering
        \includegraphics[width=.90\textwidth, alt={An EMNIST character}]{./images/issues/excluded-W-3-stroke.png}
        \caption{Stroke}
        %\label{fig:inf_issue_G0_stroke}
    \end{subfigure}%
    \begin{subfigure}{.20\columnwidth}
        \centering
        \includegraphics[width=.90\textwidth, alt={An EMNIST character}]{./images/issues/excluded-W-3-endpoint.png}
        \caption{Endpoint}
        %\label{fig:inf_issue_T5}
    \end{subfigure}%
    \begin{subfigure}{.20\columnwidth}
        \centering
        \includegraphics[width=.90\textwidth, alt={An EMNIST character}]{./images/issues/excluded-W-3-area.png}
        \caption{Area}
        %\label{fig:inf_issue_O7}
    \end{subfigure}%
    \begin{subfigure}{.20\columnwidth}
        \centering
        \includegraphics[width=.90\textwidth, alt={An EMNIST character}]{./images/issues/excluded-W-3-line.png}
        \caption{Line}
        %\label{fig:inf_issue_W3}
    \end{subfigure}%
    %\begin{subfigure}{.20\columnwidth}
    %    \centering
    %    \includegraphics[width=.90\textwidth, alt={An EMNIST character}]{./images/issues/excluded-O-7-fill.png}
    %    \caption{Enclosed Region}
    %    %\label{fig:inf_issue_O10}
    %\end{subfigure}
    \caption{Relevant transforms of Figure~\ref{fig:inf_issue_W3}.}
    \label{fig:w3_trans}
\end{figure}

Figure~\ref{fig:inf_issue_W3}, originally labeled as a W, was inferred correctly
as an M with 65\% confidence. The next closest character was a Z with 28\%
confidence.  Rationale for selecting M was consistency with the stroke,
endpoint, area, and line properties.  Relevant transforms are depicted in
Figure~\ref{fig:w3_trans}

%Figure~\ref{fig:inf_issue_O10}, originally labeled as an O was inferred correctly as an F with 40\% confidence.

The quality of the training set is a big concern with NN and XAI.  The pruning
methods presented attempt to improve the quality of the training set, to enhance
accuracy and explainability of the system. With respect to pruned samples, the
explainable architecture was a benefit in two ways, (a) it was able to correct
some problematic, pruned labels with a good degree of confidence (b) the
explainable architecture provided rationale based on explainable properties that
could help a user to confidently resolve issues with labels.

\section{Case-Based Explainable Results on Handwritten Digits}
\label{sec:case_based_handwriting_results}

The overall case-based results using an SVM for inference are shown in Table
\ref{tab:mnist_case_based_results}. The rows indicate unchanged MNIST data and
then reduced PCA data with various principal components. The 75 PC dataset
performed the best with $98.5\%$ accuracy and a rate of $97.7\%$
corresponmdence. The correspondence figure indicates that the K training
neighbors had a weight that favored the class that the inference engine selected
$97.7\%$ of the time.


\begin{table}[H]
    \renewcommand{\arraystretch}{1.3}
    %\centering
    \caption{Accuracy and correspondence of the case-based explainable architecture on MNIST.}
    \begin{center}
    %\resizebox{\columnwidth}{!}
    \end{center}
    \label{tab:mnist_case_based_results}
\end{table}

\subsection{Explainable Examples}
\label{sec:case_based_explainable_examples}

Explainable examples in this section were processed using a case-based
explainable architecture with a SVM inference engine with a radial basis function
kernel.  The MNIST data was reduced using 75 principal components.  Ten of the
nearest neighbors from the training set are retrieved in each case. Weighting
used for correspondence used the inverse square of distance.


The first example is the MNIST four from Figure \ref{fig:mnist_case_based_example1}.
Note that this is the same digit at index four of the MNIST test set as MNIST example one
the property-based example.


\begin{figure}[H]
    \centerline{\includegraphics[width=2cm, alt={An MNIST digit four}]{./images/mnist_samples/4_4_0.png}}
    \caption{MNIST example one, a digit four.}
    \label{fig:mnist_case_based_example1}
\end{figure}

The SVM predicted a digit four.  Ten nearest neighbors from training were
identified and are shown in Table \ref{tab:mnist_case_based_ex1}.  The distance,
index in the training database, and label are listed in the table.  The table
shows nine of the neighbors were also the digit four.  The last row in the table
indicates the furthest of the ten neighbors was a nine. The resulting
correspondence metric for the digit four is $0.923$ and the correspondence
metric from the class nine is $0.077$. The winning neighbor corresondence for
the digit four agrees with the SVM.

\begin{table}[H]
    \renewcommand{\arraystretch}{1.3}
    %\centering
    \caption{Ten nearest neighbors for case-based MNIST example one.}
    \begin{center}
    %\resizebox{\columnwidth}{!}{%
    \begin{tabular}{|c|c|c|}
        \hline
        Distance & \multicolumn{1}{p{3cm}|}{\centering Training\\Index}  & Label \\
        \hline
        \hline
        $3.184$ & $40094$ & $4$ \\
        \hline
        $3.357$ & $52920$ & $4$ \\
        \hline
        $3.448$ & $25760$ & $4$  \\
        \hline
        $3.517$ & $32719$ & $4$  \\
        \hline
        $3.594$ & $50609$ & $4$  \\
        \hline
        $3.793$ & $24734$ & $4$  \\
        \hline
        $3.807$ & $53519$ & $4$  \\
        \hline
        $3.809$ & $40600$ & $4$  \\
        \hline
        $3.933$ & $23306$ & $4$  \\
        \hline
        $4.128$ & $1732$ & $9$  \\
        \hline
    \end{tabular}
    %}
    \end{center}
    \label{tab:mnist_case_based_ex1}
\end{table}

Rationale provided by the explanation routine indicates the digit four is
predicted and the ten nearest neighbors agree with high correspondence of
$92.3\%$. An alternative with low correspondence $7.7\%$ is the digit nine. The
ten nearest neighbors can also be presented by retrieveing by index from the
MNIST training set.  The ten nearest neighbors to MNIST example one are shown in
Figure \ref{fig:mnist_case_based_ex1_knn}.

\begin{figure}[H]
    \centerline{\includegraphics[width=14cm, alt={The ten nearest neighbors from  training to the MNIST digit four}]{./images/case_based_mnist_ex1_knn.png}}
    \caption{Ten nearest training neighbors for MNIST example one.}
    \label{fig:mnist_case_based_ex1_knn}
\end{figure}


%%%%%%%%%%%%%%%%%%%%%%%% Example 2

The second example is shown in Figure \ref{fig:mnist_case_based_example2} and is
labeled a two. The SVM appropriately predicts a two.  The ten nearest neighbors
are shown in Table \ref{tab:mnist_case_based_ex2}.  All ten neighbors are also
labeled as the digit two.  The resulting correspondence is $1.00$ for the digit
two based on the ten nearest neighbors.

\begin{figure}[H]
    \centerline{\includegraphics[width=2cm, alt={An MNIST digit two}]{./images/mnist_samples/1_2_0.png}}
    \caption{MNIST example two, a digit two.}
    \label{fig:mnist_case_based_example2}
\end{figure}

\begin{table}[H]
    \renewcommand{\arraystretch}{1.3}
    %\centering
    \caption{Ten nearest neighbors for case-based MNIST example two.}
    \begin{center}
    %\resizebox{\columnwidth}{!}{%
    \begin{tabular}{|c|c|c|}
        \hline
        Distance & \multicolumn{1}{p{3cm}|}{\centering Training\\Index}  & Label \\
        \hline
        \hline
        $3.395$ & $28882$ & $2$ \\
        \hline
        $3.475$ & $49160$ & $2$ \\
        \hline
        $3.631$ & $24612$ & $2$  \\
        \hline
        $3.897$ & $16902$ & $2$  \\
        \hline
        $3.969$ & $31956$ & $2$  \\
        \hline
        $4.115$ & $24010$ & $2$  \\
        \hline
        $4.130$ & $31634$ & $2$  \\
        \hline
        $4.153$ & $53455$ & $2$  \\
        \hline
        $4.173$ & $17757$ & $2$  \\
        \hline
        $4.319$ & $21012$ & $2$  \\
        \hline
    \end{tabular}
    %}
    \end{center}
    \label{tab:mnist_case_based_ex2}
\end{table}

Rationale provided by the explanation routine indicates the digit two is
predicted and the ten nearest neighbors agree with high correspondence of
$100\%$. The ten nearest neighbors are also be presented by retrieveing by index
from the MNIST training set.  The ten nearest neighbors to MNIST example two are
shown in Figure \ref{fig:mnist_case_based_ex2_knn}.

\begin{figure}[H]
    \centerline{\includegraphics[width=14cm, alt={The ten nearest neighbors from training to the MNIST two}]{./images/case_based_mnist_ex2_knn.png}}
    \caption{Ten nearest training neighbors for MNIST example two.}
    \label{fig:mnist_case_based_ex2_knn}
\end{figure}


%%%%%%%%%%%%%%%%%%%%%%% example 3

The third case-based MNIST example is depicted in Figure
\ref{fig:mnist_case_based_example3}, which is labeled as a digit nine.  The ten
nearest neighbors, shown in Table \ref{tab:mnist_case_based_ex3}, are all
labeled as the digit nine as well.  the resulting correspondence from the ten
neighbors is $1.0$.

\begin{figure}[H]
    \centerline{\includegraphics[width=2cm, alt={An MNIST digit nine}]{./images/mnist_samples/7_9_0.png}}
    \caption{MNIST example three, a digit nine.}
    \label{fig:mnist_case_based_example3}
\end{figure}

\begin{table}[H]
    \renewcommand{\arraystretch}{1.3}
    %\centering
    \caption{Ten nearest neighbors for case-based MNIST example three.}
    \begin{center}
    %\resizebox{\columnwidth}{!}{%
    \begin{tabular}{|c|c|c|}
        \hline
        Distance & \multicolumn{1}{p{3cm}|}{\centering Training\\Index}  & Label \\
        \hline
        \hline
        $3.181$ & $41982$ & $9$ \\
        \hline
        $3.983$ & $25446$ & $9$ \\
        \hline
        $4.035$ & $14815$ & $9$  \\
        \hline
        $4.170$ & $56597$ & $9$  \\
        \hline
        $4.426$ & $48818$ & $9$  \\
        \hline
        $4.440$ & $34962$ & $9$  \\
        \hline
        $4.444$ & $33862$ & $9$  \\
        \hline
        $4.623$ & $56527$ & $9$  \\
        \hline
        $4.797$ & $56681$ & $9$  \\
        \hline
        $4.799$ & $11688$ & $9$  \\
        \hline
    \end{tabular}
    %}
    \end{center}
    \label{tab:mnist_case_based_ex3}
\end{table}

Rationale provided by the explanation routine indicates the digit nine is
predicted and the ten nearest neighbors agree with high correspondence of
$100\%$. The ten nearest neighbors are also be presented by retrieveing by index
from the MNIST training set.  The ten nearest neighbors to MNIST example three are
shown in Figure \ref{fig:mnist_case_based_ex3_knn}.

\begin{figure}[H]
    \centerline{\includegraphics[width=14cm, alt={Ten nearest neighbors from training to the MNIST nine}]{./images/case_based_mnist_ex3_knn.png}}
    \caption{Ten nearest training neighbors for MNIST example three.}
    \label{fig:mnist_case_based_ex3_knn}
\end{figure}


%%%%%%%%%%%%%%example 4 index 449

The next example exhibits some ambiguity. Figure
\ref{fig:mnist_case_based_example4} depicts MNSIT case-based example four,
labeled a three. The sample is index number 449 from the MNIST test set. The SVM
improperly suggested a five.

\begin{figure}[H]
    \centerline{\includegraphics[width=2cm, alt={An MNIST digit three}]{./images/mnist_samples/449_3_39.png}}
    \caption{MNIST example four, a digit three.}
    \label{fig:mnist_case_based_example4}
\end{figure}

\begin{table}[H]
    \renewcommand{\arraystretch}{1.3}
    %\centering
    \caption{Ten nearest neighbors for case-based MNIST example four.}
    \begin{center}
    %\resizebox{\columnwidth}{!}{%
    \begin{tabular}{|c|c|c|}
        \hline
        Distance & \multicolumn{1}{p{3cm}|}{\centering Training\\Index}  & Label \\
        \hline
        \hline
        $5.557$ & $47802$ & $3$ \\
        \hline
        $6.045$ & $48628$ & $3$ \\
        \hline
        $6.076$ & $48825$ & $5$  \\
        \hline
        $6.084$ & $59473$ & $5$  \\
        \hline
        $6.086$ & $47625$ & $3$  \\
        \hline
        $6.105$ & $2652$ & $3$  \\
        \hline
        $6.108$ & $35458$ & $3$  \\
        \hline
        $6.162$ & $41978$ & $3$  \\
        \hline
        $6.187$ & $52790$ & $3$  \\
        \hline
        $6.221$ & $16220$ & $3$  \\
        \hline
    \end{tabular}
    %}
    \end{center}
    \label{tab:mnist_case_based_ex4}
\end{table}

The ten nearest neighbors from the training index are shown in Table
\ref{tab:mnist_case_based_ex4}.  Two of the ten neighbors were labeled a five
while the eight others were labeled a three.  Correspondence for the three wins
at $0.802$ compared to $0.198$ for the class five. This results in the neighbor
correspondence contradicting the SVM.

Rationale provided by the explanation routine indicates that the digit five was
predicted. The ten nearest neighbors contradict the prediction with low $19.8\%$
correspondence for the digit three. The ten nearest neighbors favor of the digit
three with $80.2\%$ correspondence. The ten nearest neighbors from the training
set are sown in Figure \ref{fig:mnist_case_based_ex4_knn}.


\begin{figure}[H]
    \centerline{\includegraphics[width=14cm, alt={The ten nearest neighbors from training to the MNIST three}]{./images/case_based_mnist_ex4_knn.png}}
    \caption{Ten nearest training neighbors for MNIST example four.}
    \label{fig:mnist_case_based_ex4_knn}
\end{figure}
