\chapter{Discussion} \label{ch:discussion}

This chapter discusses the explainable methods and their results, outlining
where they perform well and the challenges that arise from the methods. In this
chapter, the property-based and case-based explainable methods will also be
compared and contrasted to another explainable AI method called LIME.

\section{Property-Based Method}

Positives of the property-based method are that the method is able to provide a
level that resembles reasoning in a human to justify decision making. The
rationale provided to the user in terms of handwritten digits related to the
properties which were geometric elements of the images.  Most of the time they
were of utility in explaining the decision.

In some cases, however, the properties and transforms did not relate well to a
particular character.  This can be observed in Figure
\ref{table:mnist_example1_ex}, in the explainability matrix for MNIST example
one, where the circle and ellipse had consistent votes for the digit zero
despite the digit clearly resembling a four.  What was reassuring about the
votes was that the effectiveness of the properties for that class was very low.

Many of the transforms result in empty or misleading output transform.  A better
strategy is required for dealing with these cases in order for the
property-based architecture to perform better.

The strategy of using a threshold to suppress low effectiveness output of an
opinion was considered and implemented.  This is similar to the way a $5\%$
threshold was used for hardware trojans in the property-based method, otherwise
all 31 opinions would be registered.

Results of applying a 5\% threshold to EMNIST uppercase letters produced
an explainable accuracy of 94.2\%. Comparing the explainable results from the EMNIST
example two in Figure \ref{fig:ex2} results in the  following:

\begin{quote}
\textit{The letter C was selected with high confidence, 1.000, and explainability 0.7676 due to the stroke, convex hull, endpoint, line, corner, ellipse-circle, and crossing properties. When selecting a letter C, 1.5\% of the time it is really a G.}
\end{quote}

Where the property-based method had significant challenges was the effort
required to implement and the performance in low dimensionality datasets. It
took significant time and analysis to discover the properties and implement
transforms.  Much time included trial and error in the parameters used for
transforms.

While the property-based architecture performed admirably with handwritten
characters, when applying to a minimal dataset, such as hardware trojans there
was little to relate the five features to provide an explanation.  The resulting
approach for trojans resulting in the combinations of the features to construct
properties seemed cumbersome.

\section{Case-Based Method}

The case-based method was advantageous in that it was easy to implement and
generalize. It took far less effort and hand-crafting than the property-based
method.  The ability of the case-based system to present like cases from
training seems reassuring but is not close to how a human would behave in
explaining how it recognizes an element like a handwritten character. 

When used for hardware trojans, the low data dimensionality of the trojan data
set showed the advantages in pre-calculating the distances to neighbors. Since
the trojan dataset has only five features and potential values of the features
were relatively low, there were many collisions (duplicates). The pre-calculated
distances could be leveraged a good deal of the time.

In the case of handwritten digits with 784 pixels, each with a range of $[0,
255]$, there were an astronomical number of possible images and no collisions in
the dataset. There was no utility in caching distance calculations. As a result,
distance calculations were worst case with every sample.  Still, in the case of
MNIST calculation was reasonable at about 360 milliseconds per sample. With a
much larger dataset, there could be challenges to scale.

\section{LIME on Handwritten Digits}

When applying LIME to MNIST digits, one has the ability to extract a mask for
the image indicating the superpixels relevant to a decision. Figure
\ref{fig:lime_mnist_masks} shows the first 20 MNIST test samples with LIME
applied. Obtaining the mask examples involved first training a CNN on MNIST and
then using LIME on the specific examples to retrieve explanation masks. The mask
can be observed in yellow segmenting the image.

\begin{figure}[H]
    \centering

    \begin{subfigure}{.19\columnwidth}
        \centering
        \includegraphics[width=.90\textwidth, alt={An MNIST digit superpixel mask}]{./images/lime/mask-0.png}
        \caption{}
        \label{fig:lime_mnist0}
    \end{subfigure}%
    \begin{subfigure}{.19\columnwidth}
        \centering
        \includegraphics[width=.90\textwidth, alt={An MNIST digit superpixel mask}]{./images/lime/mask-1.png}
        \caption{}
        \label{fig:lime_mnist1}
    \end{subfigure}%
    \begin{subfigure}{.19\columnwidth}
        \centering
        \includegraphics[width=.90\textwidth, alt={An MNIST digit superpixel mask}]{./images/lime/mask-2.png}
        \caption{}
        \label{fig:lime_mnist2}
    \end{subfigure}
    \begin{subfigure}{.19\columnwidth}
        \centering
        \includegraphics[width=.90\textwidth, alt={An MNIST digit superpixel mask}]{./images/lime/mask-3.png}
        \caption{}
        \label{fig:lime_mnist3}
    \end{subfigure}%
    \begin{subfigure}{.19\columnwidth}
        \centering
        \includegraphics[width=.90\textwidth, alt={An MNIST digit superpixel mask}]{./images/lime/mask-4.png}
        \caption{}
        \label{fig:lime_mnist4}
    \end{subfigure}

    \par\medskip

    \begin{subfigure}{.19\columnwidth}
        \centering
        \includegraphics[width=.90\textwidth, alt={An MNIST digit superpixel mask}]{./images/lime/mask-5.png}
        \caption{}
        \label{fig:lime_mnist5}
    \end{subfigure}%
    \begin{subfigure}{.19\columnwidth}
        \centering
        \includegraphics[width=.90\textwidth, alt={An MNIST digit superpixel mask}]{./images/lime/mask-6.png}
        \caption{}
        \label{fig:lime_mnsit6}
    \end{subfigure}%
    \begin{subfigure}{.19\columnwidth}
        \centering
        \includegraphics[width=.90\textwidth, alt={An MNIST digit superpixel mask}]{./images/lime/mask-7.png}
        \caption{}
        \label{fig:lime_mnist7}
    \end{subfigure}
    \begin{subfigure}{.19\columnwidth}
        \centering
        \includegraphics[width=.90\textwidth, alt={An MNIST digit superpixel mask}]{./images/lime/mask-8.png}
        \caption{}
        \label{fig:lime_mnist8}
    \end{subfigure}%
    \begin{subfigure}{.19\columnwidth}
        \centering
        \includegraphics[width=.90\textwidth, alt={An MNIST digit superpixel mask}]{./images/lime/mask-9.png}
        \caption{}
        \label{fig:lime_mnist9}
    \end{subfigure}

    \par\medskip

    \begin{subfigure}{.19\columnwidth}
        \centering
        \includegraphics[width=.90\textwidth, alt={An MNIST digit superpixel mask}]{./images/lime/mask-10.png}
        \caption{}
        \label{fig:lime_mnist10}
    \end{subfigure}%
    \begin{subfigure}{.19\columnwidth}
        \centering
        \includegraphics[width=.90\textwidth, alt={An MNIST digit superpixel mask}]{./images/lime/mask-11.png}
        \caption{}
        \label{fig:lime_mnist11}
    \end{subfigure}%
    \begin{subfigure}{.19\columnwidth}
        \centering
        \includegraphics[width=.90\textwidth, alt={An MNIST digit superpixel mask}]{./images/lime/mask-12.png}
        \caption{}
        \label{fig:lime_mnist12}
    \end{subfigure}
    \begin{subfigure}{.19\columnwidth}
        \centering
        \includegraphics[width=.90\textwidth, alt={An MNIST digit superpixel mask}]{./images/lime/mask-13.png}
        \caption{}
        \label{fig:line_skel13m}
    \end{subfigure}%
    \begin{subfigure}{.19\columnwidth}
        \centering
        \includegraphics[width=.90\textwidth, alt={An MNIST digit superpixel mask}]{./images/lime/mask-14.png}
        \caption{}
        \label{fig:lime_mnist14}
    \end{subfigure}

    \par\medskip

    \begin{subfigure}{.19\columnwidth}
        \centering
        \includegraphics[width=.90\textwidth, alt={An MNIST digit superpixel mask}]{./images/lime/mask-15.png}
        \caption{}
        \label{fig:lime_mnist15}
    \end{subfigure}%
    \begin{subfigure}{.19\columnwidth}
        \centering
        \includegraphics[width=.90\textwidth, alt={An MNIST digit superpixel mask}]{./images/lime/mask-16.png}
        \caption{}
        \label{fig:lime_mnist16}
    \end{subfigure}%
    \begin{subfigure}{.19\columnwidth}
        \centering
        \includegraphics[width=.90\textwidth, alt={An MNIST digit superpixel mask}]{./images/lime/mask-17.png}
        \caption{}
        \label{fig:lime_mnist17}
    \end{subfigure}
    \begin{subfigure}{.19\columnwidth}
        \centering
        \includegraphics[width=.90\textwidth, alt={An MNIST digit superpixel mask}]{./images/lime/mask-18.png}
        \caption{}
        \label{fig:lime_mnist18}
    \end{subfigure}%
    \begin{subfigure}{.19\columnwidth}
        \centering
        \includegraphics[width=.90\textwidth, alt={An MNIST digit superpixel mask}]{./images/lime/mask-19.png}
        \caption{}
        \label{fig:lime_mnist19}
    \end{subfigure}

    \caption{Examples of LIME masks on the first 20 MNIST test samples.}
    \label{fig:lime_mnist_masks}
\end{figure}

Samples (a), (c), (f), (k), (n) and (r) in Figure \ref{fig:lime_mnist_masks} did
not produce a mask.  It was not immediately clear why a mask was not shown for
these samples. The lack of a mask did seem to occur with relative consistently
for the digits zero, one, and seven twice in the first twenty samples. It is
also of note that another zero, sample (d), and another one, sample (o), did result
in explainable masks.

The masks appear to merely provide a segmentation of the images into superpixel
boundaries. Lime has the ability to indicate the superpixel regions that
positively and negatively influenced the classification.  Identifying the
positive and negative regions of the segmented image using LIME gave a better
understanding of what regions LIME considers important.

Figure \ref{fig:lime_pos_neg_ex0} depicts two samples from MNIST tests that did
not produce masks. The original image is on the left and the illustration of
positive and negative regions is on the right. Positive regions are shown in
pink. Figure \ref{fig:lime_pos_neg_ex0} depicts the positive and negative
regions of Figure \ref{fig:lime_mnist0}, while Figure \ref{fig:lime_pos_neg_ex2}
shows Figure \ref{fig:lime_mnist2}.  The positive and negative region images
appear to show that the entire image was important.  This was the case for all
six of the samples from Figure \ref{fig:lime_mnist_masks} without masks.

\begin{figure}[H]
    \centering

    \begin{subfigure}{0.50\columnwidth}
        \centering
        \includegraphics[width=.40\textwidth, alt={An illustration of LIME on an MNIST digit}]{./images/lime/orig-0.png}
        \includegraphics[width=.40\textwidth, alt={An illustration of LIME on an MNIST digit}]{./images/lime/pos-neg-0.png}
        \caption{}
        \label{fig:lime_pos_neg_ex0}
    \end{subfigure}%
    %\par\medskip
    \begin{subfigure}{0.50\columnwidth}
        \centering
        \includegraphics[width=.40\textwidth, alt={An illustration of LIME on an MNIST digit}]{./images/lime/orig-2.png}
        \includegraphics[width=.40\textwidth, alt={An illustration of LIME on an MNIST digit}]{./images/lime/pos-neg-2.png}
        \caption{}
        \label{fig:lime_pos_neg_ex2}
    \end{subfigure}

    \caption{LIME examples of the original digit image along with positive (pink) and negative regions for two MNIST test samples without masks.}
    \label{fig:lime_pos_neg_no_mask}
\end{figure}

Two of the masks of the four, samples (e) and (g), in Figure
\ref{fig:lime_mnist_masks} seemed to be consistent. An illustration of the
positive and negative regions of the samples of the digit four are in Figure
\ref{fig:lime_pos_neg_four_test}, where (a), (b), and (c) correspond to (e),
(g), and (t) respectively in Figure \ref{fig:lime_mnist_masks}. The
corresponding masks had the same positive region denoted by the lower left
quadrant of the image's pink shading, while the positive region in Figure
\ref{fig:lime_pos_neg_ex19} consisted of the other three quadrants.

\begin{figure}[H]
    \centering

    \begin{subfigure}{0.50\columnwidth}
        \centering
        \includegraphics[width=.40\textwidth, alt={An illustration of LIME on an MNIST digit}]{./images/lime/orig-4.png}
        \includegraphics[width=.40\textwidth, alt={An illustration of LIME on an MNIST digit}]{./images/lime/pos-neg-4.png}
        \caption{}
        \label{fig:lime_pos_neg_ex4}
    \end{subfigure}%
    %\par\medskip
    \begin{subfigure}{0.50\columnwidth}
        \centering
        \includegraphics[width=.40\textwidth, alt={An illustration of LIME on an MNIST digit}]{./images/lime/orig-6.png}
        \includegraphics[width=.40\textwidth, alt={An illustration of LIME on an MNIST digit}]{./images/lime/pos-neg-6.png}
        \caption{}
        \label{fig:lime_pos_neg_ex6}
    \end{subfigure}

    \par\medskip

    \begin{subfigure}{0.50\columnwidth}
        \centering
        \includegraphics[width=.40\textwidth, alt={An illustration of LIME on an MNIST digit}]{./images/lime/orig-19.png}
        \includegraphics[width=.40\textwidth, alt={An illustration of LIME on an MNIST digit}]{./images/lime/pos-neg-19.png}
        \caption{}
        \label{fig:lime_pos_neg_ex19}
    \end{subfigure}

    \caption{LIME examples of the original digit image along with the positive (pink) and negative regions of the digit four in the first twenty MNIST test samples.}
    \label{fig:lime_pos_neg_four_test}
\end{figure}


% \begin{figure}[H]
%     \centering

%     \begin{subfigure}{0.9\columnwidth}
%         \centering
%         \includegraphics[width=.90\textwidth, alt={An MNIST digit superpixel mask}]{./images/lime/lime_pos_neg_4.png}
%         \caption{}
%         \label{fig:lime_pos_neg_ex4}
%     \end{subfigure}%

%     \par\medskip
    
%     \begin{subfigure}{0.9\columnwidth}
%         \centering
%         \includegraphics[width=.90\textwidth, alt={An MNIST digit superpixel mask}]{./images/lime/lime_pos_neg_6.png}
%         \caption{}
%         \label{fig:lime_pos_neg_ex6}
%     \end{subfigure}

%     \par\medskip

%     \begin{subfigure}{0.9\columnwidth}
%         \centering
%         \includegraphics[width=.90\textwidth, alt={An MNIST digit superpixel mask}]{./images/lime/lime_pos_neg_19.png}
%         \caption{}
%         \label{fig:lime_pos_neg_ex19}
%     \end{subfigure}%

%     \caption{Examples of the positive and negative regions of the digits four in the first twenty MNIST test samples..}
%     \label{fig:lime_pos_neg_ex19}
% \end{figure}


When observing the several nine samples form the first 20 MNIST test samples in
the figure they had more diverse masks. The positive and negative regions are
depicted in Figure \ref{fig:lime_pos_neg_nine_test}. Sample (a) and (d) had
similar positive regions, from the digit to the left of the image. Samples (b)
and (c) again had different quadrants of the image as positive regions. 

\begin{figure}[H]
    \centering

    \begin{subfigure}{0.50\columnwidth}
        \centering
        \includegraphics[width=.40\textwidth, alt={An illustration of LIME on an MNIST digit}]{./images/lime/orig-7.png}
        \includegraphics[width=.40\textwidth, alt={An illustration of LIME on an MNIST digit}]{./images/lime/pos-neg-7.png}
        \caption{}
        \label{fig:lime_pos_neg_ex7}
    \end{subfigure}%
    %\par\medskip
    \begin{subfigure}{0.50\columnwidth}
        \centering
        \includegraphics[width=.40\textwidth, alt={An illustration of LIME on an MNIST digit}]{./images/lime/orig-9.png}
        \includegraphics[width=.40\textwidth, alt={An illustration of LIME on an MNIST digit}]{./images/lime/pos-neg-9.png}
        \caption{}
        \label{fig:lime_pos_neg_ex9}
    \end{subfigure}

    \par\medskip

    \begin{subfigure}{0.50\columnwidth}
        \centering
        \includegraphics[width=.40\textwidth, alt={An illustration of LIME on an MNIST digit}]{./images/lime/orig-12.png}
        \includegraphics[width=.40\textwidth, alt={An illustration of LIME on an MNIST digit}]{./images/lime/pos-neg-12.png}
        \caption{}
        \label{fig:lime_pos_neg_ex12}
    \end{subfigure}%
    %\par\medskip
    \begin{subfigure}{0.50\columnwidth}
        \centering
        \includegraphics[width=.40\textwidth, alt={An illustration of LIME on an MNIST digit}]{./images/lime/orig-16.png}
        \includegraphics[width=.40\textwidth, alt={An illustration of LIME on an MNIST digit}]{./images/lime/pos-neg-16.png}
        \caption{}
        \label{fig:lime_pos_neg_ex16}
    \end{subfigure}

    \caption{LIME examples of the original digit image along with the positive (pink) and negative regions of the digit nine in the first twenty MNIST test samples.}
    \label{fig:lime_pos_neg_nine_test}
\end{figure}

The work on LIME demonstrates that LIME works well to identify unanticipated
behavior of a model\cite{ribeiro2016should}. LIME is well-suited for visually
gauging the relevant regions in an application input to classification. However,
the disparity of regions reported with similar application input samples and the
cases observed where the entire image constituted the relevant region
contributing to the classification cause some concern in trusting in what is
reported from samples.

\section{Comparison of Explainable Methods to LIME}

In the case of images, LIME presents a visual explanation that highlights the
regions of the image that contributed to a decision. When applying LIME to
MNIST, some results with conflicting regions were observed as well as the region
constituting the entire application input.  Such feedback is more useful in
explaining the model behavior when one is developing an effective recognition
system rather than providing a justification to an end user of such a system.

The property based explainable architecture provides written rationale relating
to explainable properties. Results from the property-based explainable
architecture provide metrics on confidence and explainability metrics. Results
of this form are better suited for providing an explanation to an end-user and
not the developer of a recognition system like LIME.

%TODO updated examples 20 MNIST  property based results with a threshold.

The case-based explainable architecture also provides written rationale for the
decision in the form of related samples from the training set to justify the decision.
The case-based explainable architecture also provides a correspondence metric between

LIME is model agnostic and can be applied to many different recognition systems.
In a similar fashion, the case-based explainable methodology is model agnostic
and can be generalized to apply to many different models. LIME may be applied
after training and does not require the training data.  The case-based
explainable architecture, however, does require the training data.

As exhibited with the hardware trojan application, the case-based explainable
architecture does not lend to some applications and models.  Application of
the property-based method must be customized for the particular application
and the performance in explaining results is going to rely on the explainable
properties in the application.
